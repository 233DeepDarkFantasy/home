
% Default to the notebook output style

    


% Inherit from the specified cell style.




    
\documentclass[11pt]{article}

    
    
    \usepackage[T1]{fontenc}
    % Nicer default font (+ math font) than Computer Modern for most use cases
    \usepackage{mathpazo}

    % Basic figure setup, for now with no caption control since it's done
    % automatically by Pandoc (which extracts ![](path) syntax from Markdown).
    \usepackage{graphicx}
    % We will generate all images so they have a width \maxwidth. This means
    % that they will get their normal width if they fit onto the page, but
    % are scaled down if they would overflow the margins.
    \makeatletter
    \def\maxwidth{\ifdim\Gin@nat@width>\linewidth\linewidth
    \else\Gin@nat@width\fi}
    \makeatother
    \let\Oldincludegraphics\includegraphics
    % Set max figure width to be 80% of text width, for now hardcoded.
    \renewcommand{\includegraphics}[1]{\Oldincludegraphics[width=.8\maxwidth]{#1}}
    % Ensure that by default, figures have no caption (until we provide a
    % proper Figure object with a Caption API and a way to capture that
    % in the conversion process - todo).
    \usepackage{caption}
    \DeclareCaptionLabelFormat{nolabel}{}
    \captionsetup{labelformat=nolabel}

    \usepackage{adjustbox} % Used to constrain images to a maximum size 
    \usepackage{xcolor} % Allow colors to be defined
    \usepackage{enumerate} % Needed for markdown enumerations to work
    \usepackage{geometry} % Used to adjust the document margins
    \usepackage{amsmath} % Equations
    \usepackage{amssymb} % Equations
    \usepackage{textcomp} % defines textquotesingle
    % Hack from http://tex.stackexchange.com/a/47451/13684:
    \AtBeginDocument{%
        \def\PYZsq{\textquotesingle}% Upright quotes in Pygmentized code
    }
    \usepackage{upquote} % Upright quotes for verbatim code
    \usepackage{eurosym} % defines \euro
    \usepackage[mathletters]{ucs} % Extended unicode (utf-8) support
    \usepackage[utf8x]{inputenc} % Allow utf-8 characters in the tex document
    \usepackage{fancyvrb} % verbatim replacement that allows latex
    \usepackage{grffile} % extends the file name processing of package graphics 
                         % to support a larger range 
    % The hyperref package gives us a pdf with properly built
    % internal navigation ('pdf bookmarks' for the table of contents,
    % internal cross-reference links, web links for URLs, etc.)
    \usepackage{hyperref}
    \usepackage{longtable} % longtable support required by pandoc >1.10
    \usepackage{booktabs}  % table support for pandoc > 1.12.2
    \usepackage[inline]{enumitem} % IRkernel/repr support (it uses the enumerate* environment)
    \usepackage[normalem]{ulem} % ulem is needed to support strikethroughs (\sout)
                                % normalem makes italics be italics, not underlines
    

    
    
    % Colors for the hyperref package
    \definecolor{urlcolor}{rgb}{0,.145,.698}
    \definecolor{linkcolor}{rgb}{.71,0.21,0.01}
    \definecolor{citecolor}{rgb}{.12,.54,.11}

    % ANSI colors
    \definecolor{ansi-black}{HTML}{3E424D}
    \definecolor{ansi-black-intense}{HTML}{282C36}
    \definecolor{ansi-red}{HTML}{E75C58}
    \definecolor{ansi-red-intense}{HTML}{B22B31}
    \definecolor{ansi-green}{HTML}{00A250}
    \definecolor{ansi-green-intense}{HTML}{007427}
    \definecolor{ansi-yellow}{HTML}{DDB62B}
    \definecolor{ansi-yellow-intense}{HTML}{B27D12}
    \definecolor{ansi-blue}{HTML}{208FFB}
    \definecolor{ansi-blue-intense}{HTML}{0065CA}
    \definecolor{ansi-magenta}{HTML}{D160C4}
    \definecolor{ansi-magenta-intense}{HTML}{A03196}
    \definecolor{ansi-cyan}{HTML}{60C6C8}
    \definecolor{ansi-cyan-intense}{HTML}{258F8F}
    \definecolor{ansi-white}{HTML}{C5C1B4}
    \definecolor{ansi-white-intense}{HTML}{A1A6B2}

    % commands and environments needed by pandoc snippets
    % extracted from the output of `pandoc -s`
    \providecommand{\tightlist}{%
      \setlength{\itemsep}{0pt}\setlength{\parskip}{0pt}}
    \DefineVerbatimEnvironment{Highlighting}{Verbatim}{commandchars=\\\{\}}
    % Add ',fontsize=\small' for more characters per line
    \newenvironment{Shaded}{}{}
    \newcommand{\KeywordTok}[1]{\textcolor[rgb]{0.00,0.44,0.13}{\textbf{{#1}}}}
    \newcommand{\DataTypeTok}[1]{\textcolor[rgb]{0.56,0.13,0.00}{{#1}}}
    \newcommand{\DecValTok}[1]{\textcolor[rgb]{0.25,0.63,0.44}{{#1}}}
    \newcommand{\BaseNTok}[1]{\textcolor[rgb]{0.25,0.63,0.44}{{#1}}}
    \newcommand{\FloatTok}[1]{\textcolor[rgb]{0.25,0.63,0.44}{{#1}}}
    \newcommand{\CharTok}[1]{\textcolor[rgb]{0.25,0.44,0.63}{{#1}}}
    \newcommand{\StringTok}[1]{\textcolor[rgb]{0.25,0.44,0.63}{{#1}}}
    \newcommand{\CommentTok}[1]{\textcolor[rgb]{0.38,0.63,0.69}{\textit{{#1}}}}
    \newcommand{\OtherTok}[1]{\textcolor[rgb]{0.00,0.44,0.13}{{#1}}}
    \newcommand{\AlertTok}[1]{\textcolor[rgb]{1.00,0.00,0.00}{\textbf{{#1}}}}
    \newcommand{\FunctionTok}[1]{\textcolor[rgb]{0.02,0.16,0.49}{{#1}}}
    \newcommand{\RegionMarkerTok}[1]{{#1}}
    \newcommand{\ErrorTok}[1]{\textcolor[rgb]{1.00,0.00,0.00}{\textbf{{#1}}}}
    \newcommand{\NormalTok}[1]{{#1}}
    
    % Additional commands for more recent versions of Pandoc
    \newcommand{\ConstantTok}[1]{\textcolor[rgb]{0.53,0.00,0.00}{{#1}}}
    \newcommand{\SpecialCharTok}[1]{\textcolor[rgb]{0.25,0.44,0.63}{{#1}}}
    \newcommand{\VerbatimStringTok}[1]{\textcolor[rgb]{0.25,0.44,0.63}{{#1}}}
    \newcommand{\SpecialStringTok}[1]{\textcolor[rgb]{0.73,0.40,0.53}{{#1}}}
    \newcommand{\ImportTok}[1]{{#1}}
    \newcommand{\DocumentationTok}[1]{\textcolor[rgb]{0.73,0.13,0.13}{\textit{{#1}}}}
    \newcommand{\AnnotationTok}[1]{\textcolor[rgb]{0.38,0.63,0.69}{\textbf{\textit{{#1}}}}}
    \newcommand{\CommentVarTok}[1]{\textcolor[rgb]{0.38,0.63,0.69}{\textbf{\textit{{#1}}}}}
    \newcommand{\VariableTok}[1]{\textcolor[rgb]{0.10,0.09,0.49}{{#1}}}
    \newcommand{\ControlFlowTok}[1]{\textcolor[rgb]{0.00,0.44,0.13}{\textbf{{#1}}}}
    \newcommand{\OperatorTok}[1]{\textcolor[rgb]{0.40,0.40,0.40}{{#1}}}
    \newcommand{\BuiltInTok}[1]{{#1}}
    \newcommand{\ExtensionTok}[1]{{#1}}
    \newcommand{\PreprocessorTok}[1]{\textcolor[rgb]{0.74,0.48,0.00}{{#1}}}
    \newcommand{\AttributeTok}[1]{\textcolor[rgb]{0.49,0.56,0.16}{{#1}}}
    \newcommand{\InformationTok}[1]{\textcolor[rgb]{0.38,0.63,0.69}{\textbf{\textit{{#1}}}}}
    \newcommand{\WarningTok}[1]{\textcolor[rgb]{0.38,0.63,0.69}{\textbf{\textit{{#1}}}}}
    
    
    % Define a nice break command that doesn't care if a line doesn't already
    % exist.
    \def\br{\hspace*{\fill} \\* }
    % Math Jax compatability definitions
    \def\gt{>}
    \def\lt{<}
    % Document parameters
    \title{03\_SOME\_SIMPLE\_NUMERICAL\_PROGRAMS}
    
    
    

    % Pygments definitions
    
\makeatletter
\def\PY@reset{\let\PY@it=\relax \let\PY@bf=\relax%
    \let\PY@ul=\relax \let\PY@tc=\relax%
    \let\PY@bc=\relax \let\PY@ff=\relax}
\def\PY@tok#1{\csname PY@tok@#1\endcsname}
\def\PY@toks#1+{\ifx\relax#1\empty\else%
    \PY@tok{#1}\expandafter\PY@toks\fi}
\def\PY@do#1{\PY@bc{\PY@tc{\PY@ul{%
    \PY@it{\PY@bf{\PY@ff{#1}}}}}}}
\def\PY#1#2{\PY@reset\PY@toks#1+\relax+\PY@do{#2}}

\expandafter\def\csname PY@tok@mf\endcsname{\def\PY@tc##1{\textcolor[rgb]{0.40,0.40,0.40}{##1}}}
\expandafter\def\csname PY@tok@c\endcsname{\let\PY@it=\textit\def\PY@tc##1{\textcolor[rgb]{0.25,0.50,0.50}{##1}}}
\expandafter\def\csname PY@tok@gu\endcsname{\let\PY@bf=\textbf\def\PY@tc##1{\textcolor[rgb]{0.50,0.00,0.50}{##1}}}
\expandafter\def\csname PY@tok@ow\endcsname{\let\PY@bf=\textbf\def\PY@tc##1{\textcolor[rgb]{0.67,0.13,1.00}{##1}}}
\expandafter\def\csname PY@tok@vg\endcsname{\def\PY@tc##1{\textcolor[rgb]{0.10,0.09,0.49}{##1}}}
\expandafter\def\csname PY@tok@ss\endcsname{\def\PY@tc##1{\textcolor[rgb]{0.10,0.09,0.49}{##1}}}
\expandafter\def\csname PY@tok@gh\endcsname{\let\PY@bf=\textbf\def\PY@tc##1{\textcolor[rgb]{0.00,0.00,0.50}{##1}}}
\expandafter\def\csname PY@tok@dl\endcsname{\def\PY@tc##1{\textcolor[rgb]{0.73,0.13,0.13}{##1}}}
\expandafter\def\csname PY@tok@kn\endcsname{\let\PY@bf=\textbf\def\PY@tc##1{\textcolor[rgb]{0.00,0.50,0.00}{##1}}}
\expandafter\def\csname PY@tok@sc\endcsname{\def\PY@tc##1{\textcolor[rgb]{0.73,0.13,0.13}{##1}}}
\expandafter\def\csname PY@tok@gs\endcsname{\let\PY@bf=\textbf}
\expandafter\def\csname PY@tok@w\endcsname{\def\PY@tc##1{\textcolor[rgb]{0.73,0.73,0.73}{##1}}}
\expandafter\def\csname PY@tok@kp\endcsname{\def\PY@tc##1{\textcolor[rgb]{0.00,0.50,0.00}{##1}}}
\expandafter\def\csname PY@tok@m\endcsname{\def\PY@tc##1{\textcolor[rgb]{0.40,0.40,0.40}{##1}}}
\expandafter\def\csname PY@tok@sr\endcsname{\def\PY@tc##1{\textcolor[rgb]{0.73,0.40,0.53}{##1}}}
\expandafter\def\csname PY@tok@err\endcsname{\def\PY@bc##1{\setlength{\fboxsep}{0pt}\fcolorbox[rgb]{1.00,0.00,0.00}{1,1,1}{\strut ##1}}}
\expandafter\def\csname PY@tok@cm\endcsname{\let\PY@it=\textit\def\PY@tc##1{\textcolor[rgb]{0.25,0.50,0.50}{##1}}}
\expandafter\def\csname PY@tok@o\endcsname{\def\PY@tc##1{\textcolor[rgb]{0.40,0.40,0.40}{##1}}}
\expandafter\def\csname PY@tok@se\endcsname{\let\PY@bf=\textbf\def\PY@tc##1{\textcolor[rgb]{0.73,0.40,0.13}{##1}}}
\expandafter\def\csname PY@tok@si\endcsname{\let\PY@bf=\textbf\def\PY@tc##1{\textcolor[rgb]{0.73,0.40,0.53}{##1}}}
\expandafter\def\csname PY@tok@mh\endcsname{\def\PY@tc##1{\textcolor[rgb]{0.40,0.40,0.40}{##1}}}
\expandafter\def\csname PY@tok@ge\endcsname{\let\PY@it=\textit}
\expandafter\def\csname PY@tok@nc\endcsname{\let\PY@bf=\textbf\def\PY@tc##1{\textcolor[rgb]{0.00,0.00,1.00}{##1}}}
\expandafter\def\csname PY@tok@nd\endcsname{\def\PY@tc##1{\textcolor[rgb]{0.67,0.13,1.00}{##1}}}
\expandafter\def\csname PY@tok@cs\endcsname{\let\PY@it=\textit\def\PY@tc##1{\textcolor[rgb]{0.25,0.50,0.50}{##1}}}
\expandafter\def\csname PY@tok@s\endcsname{\def\PY@tc##1{\textcolor[rgb]{0.73,0.13,0.13}{##1}}}
\expandafter\def\csname PY@tok@nf\endcsname{\def\PY@tc##1{\textcolor[rgb]{0.00,0.00,1.00}{##1}}}
\expandafter\def\csname PY@tok@gr\endcsname{\def\PY@tc##1{\textcolor[rgb]{1.00,0.00,0.00}{##1}}}
\expandafter\def\csname PY@tok@kc\endcsname{\let\PY@bf=\textbf\def\PY@tc##1{\textcolor[rgb]{0.00,0.50,0.00}{##1}}}
\expandafter\def\csname PY@tok@sx\endcsname{\def\PY@tc##1{\textcolor[rgb]{0.00,0.50,0.00}{##1}}}
\expandafter\def\csname PY@tok@s2\endcsname{\def\PY@tc##1{\textcolor[rgb]{0.73,0.13,0.13}{##1}}}
\expandafter\def\csname PY@tok@ch\endcsname{\let\PY@it=\textit\def\PY@tc##1{\textcolor[rgb]{0.25,0.50,0.50}{##1}}}
\expandafter\def\csname PY@tok@sh\endcsname{\def\PY@tc##1{\textcolor[rgb]{0.73,0.13,0.13}{##1}}}
\expandafter\def\csname PY@tok@kd\endcsname{\let\PY@bf=\textbf\def\PY@tc##1{\textcolor[rgb]{0.00,0.50,0.00}{##1}}}
\expandafter\def\csname PY@tok@ni\endcsname{\let\PY@bf=\textbf\def\PY@tc##1{\textcolor[rgb]{0.60,0.60,0.60}{##1}}}
\expandafter\def\csname PY@tok@na\endcsname{\def\PY@tc##1{\textcolor[rgb]{0.49,0.56,0.16}{##1}}}
\expandafter\def\csname PY@tok@gd\endcsname{\def\PY@tc##1{\textcolor[rgb]{0.63,0.00,0.00}{##1}}}
\expandafter\def\csname PY@tok@cpf\endcsname{\let\PY@it=\textit\def\PY@tc##1{\textcolor[rgb]{0.25,0.50,0.50}{##1}}}
\expandafter\def\csname PY@tok@mo\endcsname{\def\PY@tc##1{\textcolor[rgb]{0.40,0.40,0.40}{##1}}}
\expandafter\def\csname PY@tok@vi\endcsname{\def\PY@tc##1{\textcolor[rgb]{0.10,0.09,0.49}{##1}}}
\expandafter\def\csname PY@tok@il\endcsname{\def\PY@tc##1{\textcolor[rgb]{0.40,0.40,0.40}{##1}}}
\expandafter\def\csname PY@tok@k\endcsname{\let\PY@bf=\textbf\def\PY@tc##1{\textcolor[rgb]{0.00,0.50,0.00}{##1}}}
\expandafter\def\csname PY@tok@gp\endcsname{\let\PY@bf=\textbf\def\PY@tc##1{\textcolor[rgb]{0.00,0.00,0.50}{##1}}}
\expandafter\def\csname PY@tok@vm\endcsname{\def\PY@tc##1{\textcolor[rgb]{0.10,0.09,0.49}{##1}}}
\expandafter\def\csname PY@tok@nb\endcsname{\def\PY@tc##1{\textcolor[rgb]{0.00,0.50,0.00}{##1}}}
\expandafter\def\csname PY@tok@s1\endcsname{\def\PY@tc##1{\textcolor[rgb]{0.73,0.13,0.13}{##1}}}
\expandafter\def\csname PY@tok@no\endcsname{\def\PY@tc##1{\textcolor[rgb]{0.53,0.00,0.00}{##1}}}
\expandafter\def\csname PY@tok@go\endcsname{\def\PY@tc##1{\textcolor[rgb]{0.53,0.53,0.53}{##1}}}
\expandafter\def\csname PY@tok@mi\endcsname{\def\PY@tc##1{\textcolor[rgb]{0.40,0.40,0.40}{##1}}}
\expandafter\def\csname PY@tok@ne\endcsname{\let\PY@bf=\textbf\def\PY@tc##1{\textcolor[rgb]{0.82,0.25,0.23}{##1}}}
\expandafter\def\csname PY@tok@nn\endcsname{\let\PY@bf=\textbf\def\PY@tc##1{\textcolor[rgb]{0.00,0.00,1.00}{##1}}}
\expandafter\def\csname PY@tok@kr\endcsname{\let\PY@bf=\textbf\def\PY@tc##1{\textcolor[rgb]{0.00,0.50,0.00}{##1}}}
\expandafter\def\csname PY@tok@fm\endcsname{\def\PY@tc##1{\textcolor[rgb]{0.00,0.00,1.00}{##1}}}
\expandafter\def\csname PY@tok@nl\endcsname{\def\PY@tc##1{\textcolor[rgb]{0.63,0.63,0.00}{##1}}}
\expandafter\def\csname PY@tok@sb\endcsname{\def\PY@tc##1{\textcolor[rgb]{0.73,0.13,0.13}{##1}}}
\expandafter\def\csname PY@tok@sa\endcsname{\def\PY@tc##1{\textcolor[rgb]{0.73,0.13,0.13}{##1}}}
\expandafter\def\csname PY@tok@bp\endcsname{\def\PY@tc##1{\textcolor[rgb]{0.00,0.50,0.00}{##1}}}
\expandafter\def\csname PY@tok@nv\endcsname{\def\PY@tc##1{\textcolor[rgb]{0.10,0.09,0.49}{##1}}}
\expandafter\def\csname PY@tok@nt\endcsname{\let\PY@bf=\textbf\def\PY@tc##1{\textcolor[rgb]{0.00,0.50,0.00}{##1}}}
\expandafter\def\csname PY@tok@sd\endcsname{\let\PY@it=\textit\def\PY@tc##1{\textcolor[rgb]{0.73,0.13,0.13}{##1}}}
\expandafter\def\csname PY@tok@kt\endcsname{\def\PY@tc##1{\textcolor[rgb]{0.69,0.00,0.25}{##1}}}
\expandafter\def\csname PY@tok@gt\endcsname{\def\PY@tc##1{\textcolor[rgb]{0.00,0.27,0.87}{##1}}}
\expandafter\def\csname PY@tok@mb\endcsname{\def\PY@tc##1{\textcolor[rgb]{0.40,0.40,0.40}{##1}}}
\expandafter\def\csname PY@tok@cp\endcsname{\def\PY@tc##1{\textcolor[rgb]{0.74,0.48,0.00}{##1}}}
\expandafter\def\csname PY@tok@c1\endcsname{\let\PY@it=\textit\def\PY@tc##1{\textcolor[rgb]{0.25,0.50,0.50}{##1}}}
\expandafter\def\csname PY@tok@vc\endcsname{\def\PY@tc##1{\textcolor[rgb]{0.10,0.09,0.49}{##1}}}
\expandafter\def\csname PY@tok@gi\endcsname{\def\PY@tc##1{\textcolor[rgb]{0.00,0.63,0.00}{##1}}}

\def\PYZbs{\char`\\}
\def\PYZus{\char`\_}
\def\PYZob{\char`\{}
\def\PYZcb{\char`\}}
\def\PYZca{\char`\^}
\def\PYZam{\char`\&}
\def\PYZlt{\char`\<}
\def\PYZgt{\char`\>}
\def\PYZsh{\char`\#}
\def\PYZpc{\char`\%}
\def\PYZdl{\char`\$}
\def\PYZhy{\char`\-}
\def\PYZsq{\char`\'}
\def\PYZdq{\char`\"}
\def\PYZti{\char`\~}
% for compatibility with earlier versions
\def\PYZat{@}
\def\PYZlb{[}
\def\PYZrb{]}
\makeatother


    % Exact colors from NB
    \definecolor{incolor}{rgb}{0.0, 0.0, 0.5}
    \definecolor{outcolor}{rgb}{0.545, 0.0, 0.0}



    
    % Prevent overflowing lines due to hard-to-break entities
    \sloppy 
    % Setup hyperref package
    \hypersetup{
      breaklinks=true,  % so long urls are correctly broken across lines
      colorlinks=true,
      urlcolor=urlcolor,
      linkcolor=linkcolor,
      citecolor=citecolor,
      }
    % Slightly bigger margins than the latex defaults
    
    \geometry{verbose,tmargin=1in,bmargin=1in,lmargin=1in,rmargin=1in}
    
    

    \begin{document}
    
    
    \maketitle
    
    

    
    \hypertarget{some-simple-numerical-programs}{%
\section{3 SOME SIMPLE NUMERICAL
PROGRAMS}\label{some-simple-numerical-programs}}

Now that we have covered some basic Python constructs,

it is time to start thinking about how we can combine those constructs
to write some simple programs.

Along the way, we'll sneak in a few more language constructs and some
algorithmic techniques.

    \hypertarget{exhaustive-enumeration}{%
\subsection{3.1 Exhaustive Enumeration}\label{exhaustive-enumeration}}

    \begin{Verbatim}[commandchars=\\\{\}]
{\color{incolor}In [{\color{incolor}1}]:} \PY{c+c1}{\PYZsh{}Page 21, Figure 3.1}
        \PY{c+c1}{\PYZsh{}Find the cube root of a perfect cube}
        
        \PY{c+c1}{\PYZsh{}x = int(input(\PYZsq{}Enter an integer: \PYZsq{}))}
        \PY{c+c1}{\PYZsh{}x=19}
        \PY{n}{x}\PY{o}{=}\PY{l+m+mi}{18}
        \PY{c+c1}{\PYZsh{}x=\PYZhy{}8}
        
        \PY{n}{ans} \PY{o}{=} \PY{l+m+mi}{0}   \PY{c+c1}{\PYZsh{}  !!!}
        \PY{k}{while} \PY{n}{ans}\PY{o}{*}\PY{o}{*}\PY{l+m+mi}{3} \PY{o}{\PYZlt{}} \PY{n+nb}{abs}\PY{p}{(}\PY{n}{x}\PY{p}{)}\PY{p}{:}
            \PY{n}{ans} \PY{o}{=} \PY{n}{ans} \PY{o}{+} \PY{l+m+mi}{1}  \PY{c+c1}{\PYZsh{} +1  Exhaustive Enumeration}
        
        \PY{k}{if} \PY{n}{ans}\PY{o}{*}\PY{o}{*}\PY{l+m+mi}{3} \PY{o}{!=} \PY{n+nb}{abs}\PY{p}{(}\PY{n}{x}\PY{p}{)}\PY{p}{:}
            \PY{n+nb}{print}\PY{p}{(}\PY{n}{x}\PY{p}{,} \PY{l+s+s1}{\PYZsq{}}\PY{l+s+s1}{is not a perfect cube}\PY{l+s+s1}{\PYZsq{}}\PY{p}{)}
        \PY{k}{else}\PY{p}{:}
            \PY{k}{if} \PY{n}{x} \PY{o}{\PYZlt{}} \PY{l+m+mi}{0}\PY{p}{:}
                \PY{n}{ans} \PY{o}{=} \PY{o}{\PYZhy{}}\PY{n}{ans}
            \PY{n+nb}{print}\PY{p}{(}\PY{l+s+s1}{\PYZsq{}}\PY{l+s+s1}{Cube root of}\PY{l+s+s1}{\PYZsq{}}\PY{p}{,} \PY{n}{x}\PY{p}{,}\PY{l+s+s1}{\PYZsq{}}\PY{l+s+s1}{is}\PY{l+s+s1}{\PYZsq{}}\PY{p}{,} \PY{n}{ans}\PY{p}{)}
\end{Verbatim}


    \begin{Verbatim}[commandchars=\\\{\}]
18 is not a perfect cube

    \end{Verbatim}

    The algorithmic technique used in this program is a variant of guess and
check called exhaustive enumeration.

We enumerate all possibilities until we get to the right answer or
exhaust the space of possibilities.

At first blush, this may seem like an incredibly stupid way to solve a
problem.

Surprisingly, however, exhaustive enumeration algorithms are often the
most practical way to solve a problem.

    \begin{Verbatim}[commandchars=\\\{\}]
{\color{incolor}In [{\color{incolor} }]:} \PY{n}{x}\PY{o}{=}\PY{l+m+mi}{1957816251}  \PY{c+c1}{\PYZsh{} 7406961012236344616 very big }
        \PY{n}{ans} \PY{o}{=} \PY{l+m+mi}{0}   
        \PY{k}{while} \PY{n}{ans}\PY{o}{*}\PY{o}{*}\PY{l+m+mi}{3} \PY{o}{\PYZlt{}} \PY{n+nb}{abs}\PY{p}{(}\PY{n}{x}\PY{p}{)}\PY{p}{:}
            \PY{n}{ans} \PY{o}{=} \PY{n}{ans} \PY{o}{+} \PY{l+m+mi}{1}  \PY{c+c1}{\PYZsh{} }
        
        \PY{k}{if} \PY{n}{ans}\PY{o}{*}\PY{o}{*}\PY{l+m+mi}{3} \PY{o}{!=} \PY{n+nb}{abs}\PY{p}{(}\PY{n}{x}\PY{p}{)}\PY{p}{:}
            \PY{n+nb}{print}\PY{p}{(}\PY{n}{x}\PY{p}{,} \PY{l+s+s1}{\PYZsq{}}\PY{l+s+s1}{is not a perfect cube}\PY{l+s+s1}{\PYZsq{}}\PY{p}{)}
        \PY{k}{else}\PY{p}{:}
            \PY{k}{if} \PY{n}{x} \PY{o}{\PYZlt{}} \PY{l+m+mi}{0}\PY{p}{:}
                \PY{n}{ans} \PY{o}{=} \PY{o}{\PYZhy{}}\PY{n}{ans}
            \PY{n+nb}{print}\PY{p}{(}\PY{l+s+s1}{\PYZsq{}}\PY{l+s+s1}{Cube root of}\PY{l+s+s1}{\PYZsq{}}\PY{p}{,} \PY{n}{x}\PY{p}{,}\PY{l+s+s1}{\PYZsq{}}\PY{l+s+s1}{is}\PY{l+s+s1}{\PYZsq{}}\PY{p}{,} \PY{n}{ans}\PY{p}{)}
\end{Verbatim}


    \hypertarget{now-lets-insert-some-errors-and-see-what-happens}{%
\subsubsection{Now, let's insert some errors and see what
happens}\label{now-lets-insert-some-errors-and-see-what-happens}}

\begin{itemize}
\tightlist
\item
  1 commenting out the statement ans = 0.
\end{itemize}

The Python interpreter prints the error message, NameError: name `ans'
is not defined, because the interpreter attempts to find the value to
which ans is bound before it has been bound to anything.

    \begin{Verbatim}[commandchars=\\\{\}]
{\color{incolor}In [{\color{incolor} }]:} \PY{n}{x}\PY{o}{=}\PY{o}{\PYZhy{}}\PY{l+m+mi}{8}
        
        \PY{c+c1}{\PYZsh{} ans = 0   \PYZsh{}  commenting out the statement ans = 0}
                    \PY{c+c1}{\PYZsh{}              2.When the loop is entered, its value is nonnegative}
        \PY{k}{while} \PY{n}{ans}\PY{o}{*}\PY{o}{*}\PY{l+m+mi}{3} \PY{o}{\PYZlt{}} \PY{n+nb}{abs}\PY{p}{(}\PY{n}{x}\PY{p}{)}\PY{p}{:}
            \PY{n}{ans} \PY{o}{=} \PY{n}{ans} \PY{o}{+} \PY{l+m+mi}{1}  \PY{c+c1}{\PYZsh{} }
        
        \PY{k}{if} \PY{n}{ans}\PY{o}{*}\PY{o}{*}\PY{l+m+mi}{3} \PY{o}{!=} \PY{n+nb}{abs}\PY{p}{(}\PY{n}{x}\PY{p}{)}\PY{p}{:}
            \PY{n+nb}{print}\PY{p}{(}\PY{n}{x}\PY{p}{,} \PY{l+s+s1}{\PYZsq{}}\PY{l+s+s1}{is not a perfect cube}\PY{l+s+s1}{\PYZsq{}}\PY{p}{)}
        \PY{k}{else}\PY{p}{:}
            \PY{k}{if} \PY{n}{x} \PY{o}{\PYZlt{}} \PY{l+m+mi}{0}\PY{p}{:}
                \PY{n}{ans} \PY{o}{=} \PY{o}{\PYZhy{}}\PY{n}{ans}
            \PY{n+nb}{print}\PY{p}{(}\PY{l+s+s1}{\PYZsq{}}\PY{l+s+s1}{Cube root of}\PY{l+s+s1}{\PYZsq{}}\PY{p}{,} \PY{n}{x}\PY{p}{,}\PY{l+s+s1}{\PYZsq{}}\PY{l+s+s1}{is}\PY{l+s+s1}{\PYZsq{}}\PY{p}{,} \PY{n}{ans}\PY{p}{)}
\end{Verbatim}


    \begin{itemize}
\tightlist
\item
  2 replace the statement ans = ans + 1 by ans = ans,
\end{itemize}

and try finding the cube root of 8. After you get tired of waiting,
enter ``control+c'' (hold down the control key and the c key
simultaneously). This will return you to the user prompt in the shell.

    \begin{Verbatim}[commandchars=\\\{\}]
{\color{incolor}In [{\color{incolor} }]:} \PY{o}{\PYZpc{}\PYZpc{}}\PY{k}{file} ./src/ch3\PYZus{}cube\PYZus{}root.py
        \PYZsh{} Test in IDEL By ch3\PYZus{}cube\PYZus{}root
        x=8
        ans = 0  
        while ans**3 \PYZlt{} abs(x):
            
            ans = ans  \PYZsh{} replace the statement ans = ans + 1 by ans = ans 
                       \PYZsh{} 4.Its value is decreased every time through the loop. 
                       \PYZsh{} 3.When its value is \PYZlt{}=0, the loop terminates. 
        
        if ans**3 != abs(x):
            print(x, \PYZsq{}is not a perfect cube\PYZsq{})
        else:
            if x \PYZlt{} 0:
                ans = \PYZhy{}ans
            print(\PYZsq{}Cube root of\PYZsq{}, x,\PYZsq{}is\PYZsq{}, ans)
\end{Verbatim}


    \begin{Verbatim}[commandchars=\\\{\}]
{\color{incolor}In [{\color{incolor} }]:} \PY{o}{!}python ./src/ch3\PYZus{}cube\PYZus{}root.py
\end{Verbatim}


    \hypertarget{when-confronted-with-a-program-that-seems-not-to-be-terminating}{%
\subsubsection{When confronted with a program that seems not to be
terminating,}\label{when-confronted-with-a-program-that-seems-not-to-be-terminating}}

insert print statements to test whether the decrementing function is
indeed being decremented.

    \hypertarget{decrementing-function}{%
\subsubsection{decrementing function}\label{decrementing-function}}

Whenever you write a loop, you should think about an appropriate
decrementing function. This is a function that has the following
properties: * It maps a set of program variables into an integer. * When
the loop is entered, its value is nonnegative. * When its value is
\textless{}=0, the loop \textbf{terminates}. * Its value is
\textbf{decreased} every time through the loop.

What is the decrementing function for the loop? It is

\begin{Shaded}
\begin{Highlighting}[]
\BuiltInTok{abs}\NormalTok{(x) }\OperatorTok{-}\NormalTok{ ans}\OperatorTok{**}\DecValTok{3}
\end{Highlighting}
\end{Shaded}

    \begin{Verbatim}[commandchars=\\\{\}]
{\color{incolor}In [{\color{incolor} }]:} \PY{o}{\PYZpc{}\PYZpc{}}\PY{k}{file} ./src/ch3\PYZus{}cube\PYZus{}root.py
        
        \PYZsh{} Test in IDEL By ch3\PYZus{}cube\PYZus{}root
        x=8
        ans = 0  
        while ans**3 \PYZlt{} abs(x):
            
            print(\PYZsq{}Value of the decrementing function abs(x) \PYZhy{} ans**3 is\PYZsq{}, 
                    abs(x) \PYZhy{} ans**3) \PYZsh{} add the statement at the start of the loop
                                     \PYZsh{} test whether the decrementing function is indeed being decremented
           
            ans = ans  \PYZsh{} replace the statement ans = ans + 1 by ans = ans 
                       \PYZsh{} 4.Its value is decreased every time through the loop. 
                       \PYZsh{} 3.When its value is \PYZlt{}=0, the loop terminates. 
        
        if ans**3 != abs(x):
            print(x, \PYZsq{}is not a perfect cube\PYZsq{})
        else:
            if x \PYZlt{} 0:
                ans = \PYZhy{}ans
            print(\PYZsq{}Cube root of\PYZsq{}, x,\PYZsq{}is\PYZsq{}, ans)
\end{Verbatim}


    \begin{Verbatim}[commandchars=\\\{\}]
{\color{incolor}In [{\color{incolor} }]:} \PY{o}{!}python ./src/ch3\PYZus{}cube\PYZus{}root.py
\end{Verbatim}


    \hypertarget{line-continuation.}{%
\subsubsection{line continuation.}\label{line-continuation.}}

\begin{itemize}
\item
  Python's implicit line joining inside parentheses, brackets and
  braces.
\item
  ``" continuation.
\end{itemize}

    \begin{Verbatim}[commandchars=\\\{\}]
{\color{incolor}In [{\color{incolor} }]:} \PY{n}{x}\PY{o}{=}\PY{l+m+mi}{1}
        \PY{n}{ans}\PY{o}{=}\PY{l+m+mi}{2}
        \PY{n+nb}{print}\PY{p}{(}\PY{l+s+s1}{\PYZsq{}}\PY{l+s+s1}{Value of the decrementing function abs(x) \PYZhy{} ans**3 is}\PY{l+s+s1}{\PYZsq{}}\PY{p}{,} 
                    \PY{n+nb}{abs}\PY{p}{(}\PY{n}{x}\PY{p}{)} \PY{o}{\PYZhy{}} \PY{n}{ans}\PY{o}{*}\PY{o}{*}\PY{l+m+mi}{3}\PY{p}{)}
\end{Verbatim}


    \begin{Verbatim}[commandchars=\\\{\}]
{\color{incolor}In [{\color{incolor} }]:} \PY{n}{a} \PY{o}{=} \PY{l+s+s1}{\PYZsq{}}\PY{l+s+s1}{1}\PY{l+s+s1}{\PYZsq{}} \PY{o}{+} \PY{l+s+s1}{\PYZsq{}}\PY{l+s+s1}{2}\PY{l+s+s1}{\PYZsq{}} \PY{o}{+} \PY{l+s+s1}{\PYZsq{}}\PY{l+s+s1}{3}\PY{l+s+s1}{\PYZsq{}} \PY{o}{+} \PYZbs{}
            \PY{l+s+s1}{\PYZsq{}}\PY{l+s+s1}{4}\PY{l+s+s1}{\PYZsq{}} \PY{o}{+} \PY{l+s+s1}{\PYZsq{}}\PY{l+s+s1}{5}\PY{l+s+s1}{\PYZsq{}}
        \PY{n}{a}
\end{Verbatim}


    \begin{Verbatim}[commandchars=\\\{\}]
{\color{incolor}In [{\color{incolor} }]:} \PY{n}{a} \PY{o}{=} \PY{p}{(}\PY{l+s+s1}{\PYZsq{}}\PY{l+s+s1}{1}\PY{l+s+s1}{\PYZsq{}} \PY{o}{+} \PY{l+s+s1}{\PYZsq{}}\PY{l+s+s1}{2}\PY{l+s+s1}{\PYZsq{}} \PY{o}{+} \PY{l+s+s1}{\PYZsq{}}\PY{l+s+s1}{3}\PY{l+s+s1}{\PYZsq{}} \PY{o}{+}
            \PY{l+s+s1}{\PYZsq{}}\PY{l+s+s1}{4}\PY{l+s+s1}{\PYZsq{}} \PY{o}{+} \PY{l+s+s1}{\PYZsq{}}\PY{l+s+s1}{5}\PY{l+s+s1}{\PYZsq{}}\PY{p}{)}
        \PY{n}{a}
\end{Verbatim}


    \hypertarget{dynamic-typing}{%
\subsubsection{Dynamic typing}\label{dynamic-typing}}

Python is dynamically typed, which means that variables do not have a
fixed type.

In fact, in Python, variables are very different from what they are in
many other languages, specifically statically-typed languages.

\textbf{Variables are not a segment of the computer's memory where some
value is written, they are `tags' or `names' pointing to objects}.

It is therefore possible for the variable `a' to be set to the value 1,
then to the value `a string', then to a function.

    \begin{Verbatim}[commandchars=\\\{\}]
{\color{incolor}In [{\color{incolor} }]:} \PY{n}{a} \PY{o}{=} \PY{l+m+mi}{1}   \PY{c+c1}{\PYZsh{} variables do not have a fixed type,just names pointing to objects}
        \PY{n+nb}{print}\PY{p}{(}\PY{n}{a}\PY{p}{)}
        \PY{n}{a} \PY{o}{=} \PY{l+s+s1}{\PYZsq{}}\PY{l+s+s1}{a string}\PY{l+s+s1}{\PYZsq{}}
        \PY{n+nb}{print}\PY{p}{(}\PY{n}{a}\PY{p}{)}
        \PY{k}{def} \PY{n+nf}{a}\PY{p}{(}\PY{p}{)}\PY{p}{:}
            \PY{k}{pass}  \PY{c+c1}{\PYZsh{} Do something}
        \PY{n+nb}{print}\PY{p}{(}\PY{n}{a}\PY{p}{)}
\end{Verbatim}


    The dynamic typing of Python is often considered to be a weakness, and
indeed it can lead to complexities and hard-to-debug code.

Something named \textbf{`a'} can be set to many different things, and
the developer or the maintainer needs to track this name in the code to
make sure it has not been set to a completely unrelated object.

    \hypertarget{avoid-using-the-same-variable-name-for-different-things.}{%
\subsubsection{Avoid using the same variable name for different
things.}\label{avoid-using-the-same-variable-name-for-different-things.}}

    \begin{Verbatim}[commandchars=\\\{\}]
{\color{incolor}In [{\color{incolor} }]:} \PY{n}{count} \PY{o}{=} \PY{l+m+mi}{1}
        \PY{n}{msg} \PY{o}{=} \PY{l+s+s1}{\PYZsq{}}\PY{l+s+s1}{a string}\PY{l+s+s1}{\PYZsq{}}
        \PY{k}{def} \PY{n+nf}{func}\PY{p}{(}\PY{p}{)}\PY{p}{:}
            \PY{k}{pass}  \PY{c+c1}{\PYZsh{} Do something}
\end{Verbatim}


    \hypertarget{further-reading}{%
\subsubsection{Further Reading}\label{further-reading}}

\begin{itemize}
\item
  26.1. typing --- Support for type hints New in version 3.5.

  https://docs.python.org/3.5/library/typing.html
\end{itemize}

    \hypertarget{for-loops}{%
\subsection{3.2 For Loops}\label{for-loops}}

Python provides a language mechanism, the for loop,that can be used to
simplify programs containing this kind of iteration: Each iterates over
a sequence of integers.

The general form of a for statement is :

\begin{Shaded}
\begin{Highlighting}[]
\ControlFlowTok{for}\NormalTok{ variable }\KeywordTok{in}\NormalTok{ sequence:}
\NormalTok{    code block}
\end{Highlighting}
\end{Shaded}

    The process continues until the sequence is exhausted or a break
statement is executed within the code block.

The sequence of values bound to variable is most commonly generated
using the built-in function range, which returns a sequence containing
an arithmetic progression. The range function takes three integer
arguments: start, stop, and step.

range(start,stop,step)

It produces the progression start, start + step, start + 2*step, etc.

\begin{verbatim}
range(5,40,10) -> [5,15,25,35]
\end{verbatim}

If step is negative, the last element is the smallest integer start +
i*step greater than stop.

\begin{verbatim}
range(40,5,-10)-> [40,30,20,10]
\end{verbatim}

If the first argument is omitted it defaults to 0, if the lastargument
(the step size) is omitted it defaults to 1.

\begin{verbatim}
range(3)-> range(0, 3)->range(0, 3,1) ->[0, 1, 2]
\end{verbatim}

    \begin{Verbatim}[commandchars=\\\{\}]
{\color{incolor}In [{\color{incolor} }]:} \PY{n}{x} \PY{o}{=} \PY{l+m+mi}{4}
        \PY{k}{for} \PY{n}{i} \PY{o+ow}{in} \PY{n+nb}{range}\PY{p}{(}\PY{l+m+mi}{0}\PY{p}{,} \PY{n}{x}\PY{p}{)}\PY{p}{:}
            \PY{n+nb}{print}\PY{p}{(}\PY{n}{i}\PY{p}{)} 
\end{Verbatim}


    \begin{Verbatim}[commandchars=\\\{\}]
{\color{incolor}In [{\color{incolor} }]:} \PY{n}{x} \PY{o}{=} \PY{l+m+mi}{4}
        \PY{k}{for} \PY{n}{i} \PY{o+ow}{in} \PY{n+nb}{range}\PY{p}{(}\PY{l+m+mi}{0}\PY{p}{,} \PY{n}{x}\PY{p}{)}\PY{p}{:}
            \PY{n}{x}\PY{o}{=}\PY{l+m+mi}{5}
            \PY{n+nb}{print}\PY{p}{(}\PY{n}{i}\PY{p}{)} 
\end{Verbatim}


    The range function in the line with for is evaluated just before the
first iteration of the loop, and not reevaluated for subsequent
iterations.

    \begin{Verbatim}[commandchars=\\\{\}]
{\color{incolor}In [{\color{incolor} }]:} \PY{n}{x} \PY{o}{=} \PY{l+m+mi}{4}
        \PY{k}{for} \PY{n}{j} \PY{o+ow}{in} \PY{n+nb}{range}\PY{p}{(}\PY{n}{x}\PY{p}{)}\PY{p}{:}
            \PY{n+nb}{print}\PY{p}{(}\PY{l+s+s1}{\PYZsq{}}\PY{l+s+s1}{j: }\PY{l+s+s1}{\PYZsq{}}\PY{p}{,}\PY{n}{j}\PY{p}{)}
            \PY{k}{for} \PY{n}{i} \PY{o+ow}{in} \PY{n+nb}{range}\PY{p}{(}\PY{n}{x}\PY{p}{)}\PY{p}{:}  \PY{c+c1}{\PYZsh{} inner loop}
                \PY{n+nb}{print}\PY{p}{(}\PY{n}{i}\PY{p}{)}
                \PY{n}{x}\PY{o}{=}\PY{l+m+mi}{2}              \PY{c+c1}{\PYZsh{} inner loop    }
\end{Verbatim}


    \begin{Verbatim}[commandchars=\\\{\}]
{\color{incolor}In [{\color{incolor} }]:} \PY{n}{x} \PY{o}{=} \PY{l+m+mi}{4}
        \PY{k}{for} \PY{n}{j} \PY{o+ow}{in} \PY{n+nb}{range}\PY{p}{(}\PY{n}{x}\PY{p}{)}\PY{p}{:}
            \PY{n+nb}{print}\PY{p}{(}\PY{l+s+s1}{\PYZsq{}}\PY{l+s+s1}{j: }\PY{l+s+s1}{\PYZsq{}}\PY{p}{,}\PY{n}{j}\PY{p}{)}
            \PY{k}{for} \PY{n}{i} \PY{o+ow}{in} \PY{n+nb}{range}\PY{p}{(}\PY{n}{x}\PY{p}{)}\PY{p}{:}  \PY{c+c1}{\PYZsh{} inner loop}
                \PY{n+nb}{print}\PY{p}{(}\PY{n}{i}\PY{p}{)}
            \PY{n}{x}\PY{o}{=}\PY{l+m+mi}{2} \PY{c+c1}{\PYZsh{} change  x=2}
\end{Verbatim}


    finding cube roots: while loop -\textgreater{} for loop

The break statement in the for loop causes the loop to terminate before
it has been run on each element in the sequence over which it is
iterating.

    \begin{Verbatim}[commandchars=\\\{\}]
{\color{incolor}In [{\color{incolor} }]:} \PY{c+c1}{\PYZsh{} Find the cube root of a perfect cube}
        \PY{n}{x} \PY{o}{=} \PY{n+nb}{int}\PY{p}{(}\PY{n+nb}{input}\PY{p}{(}\PY{l+s+s1}{\PYZsq{}}\PY{l+s+s1}{Enter an integer: }\PY{l+s+s1}{\PYZsq{}}\PY{p}{)}\PY{p}{)}  \PY{c+c1}{\PYZsh{}  27}
        \PY{k}{for} \PY{n}{ans} \PY{o+ow}{in} \PY{n+nb}{range}\PY{p}{(}\PY{l+m+mi}{0}\PY{p}{,} \PY{n+nb}{abs}\PY{p}{(}\PY{n}{x}\PY{p}{)} \PY{o}{+} \PY{l+m+mi}{1}\PY{p}{)}\PY{p}{:}   
            \PY{k}{if} \PY{n}{ans}\PY{o}{*}\PY{o}{*}\PY{l+m+mi}{3} \PY{o}{\PYZgt{}}\PY{o}{=} \PY{n+nb}{abs}\PY{p}{(}\PY{n}{x}\PY{p}{)}\PY{p}{:}
                \PY{k}{break}          \PY{c+c1}{\PYZsh{}  break statement}
        
        \PY{k}{if} \PY{n}{ans}\PY{o}{*}\PY{o}{*}\PY{l+m+mi}{3} \PY{o}{!=} \PY{n+nb}{abs}\PY{p}{(}\PY{n}{x}\PY{p}{)}\PY{p}{:}
            \PY{n+nb}{print}\PY{p}{(}\PY{n}{x}\PY{p}{,} \PY{l+s+s1}{\PYZsq{}}\PY{l+s+s1}{is not a perfect cube}\PY{l+s+s1}{\PYZsq{}}\PY{p}{)}
        \PY{k}{else}\PY{p}{:}
            \PY{k}{if} \PY{n}{x} \PY{o}{\PYZlt{}} \PY{l+m+mi}{0}\PY{p}{:}
                \PY{n}{ans} \PY{o}{=} \PY{o}{\PYZhy{}}\PY{n}{ans}
            \PY{n+nb}{print}\PY{p}{(}\PY{l+s+s1}{\PYZsq{}}\PY{l+s+s1}{Cube root of}\PY{l+s+s1}{\PYZsq{}}\PY{p}{,} \PY{n}{x}\PY{p}{,} \PY{l+s+s1}{\PYZsq{}}\PY{l+s+s1}{is}\PY{l+s+s1}{\PYZsq{}}\PY{p}{,} \PY{n}{ans}\PY{p}{)}
\end{Verbatim}


    The for statement can be used to conveniently iterate over characters of
a string.

    Most of the time, numbers of type float provide a reasonably good
approximation to real numbers. But ``most of the time'' is not all of
the time, and when they don't it can lead to surprising consequences.

    \begin{Verbatim}[commandchars=\\\{\}]
{\color{incolor}In [{\color{incolor} }]:} \PY{n}{total} \PY{o}{=} \PY{l+m+mi}{0}
        \PY{k}{for} \PY{n}{c} \PY{o+ow}{in} \PY{l+s+s1}{\PYZsq{}}\PY{l+s+s1}{123456789}\PY{l+s+s1}{\PYZsq{}}\PY{p}{:}
            \PY{n}{total} \PY{o}{=} \PY{n}{total} \PY{o}{+} \PY{n+nb}{int}\PY{p}{(}\PY{n}{c}\PY{p}{)}
        \PY{n+nb}{print}\PY{p}{(}\PY{n}{total}\PY{p}{)}
\end{Verbatim}


    \hypertarget{approximate-solutions-and-bisection-search}{%
\subsection{3.3 Approximate Solutions and Bisection
Search}\label{approximate-solutions-and-bisection-search}}

Imagine that someone asks you to write a program that finds the square
root of any nonnegative number. What should you do?

The right thing to have asked for is a program that finds an
approximation to the square root---i.e., an answer that is close enough
to the actual square root to be useful.

numerical solution

analytical solution

    \hypertarget{approximating-the-square-root-using-exhaustive-enumeration}{%
\subsubsection{1) Approximating the square root using exhaustive
enumeration}\label{approximating-the-square-root-using-exhaustive-enumeration}}

    \begin{Verbatim}[commandchars=\\\{\}]
{\color{incolor}In [{\color{incolor} }]:} \PY{c+c1}{\PYZsh{}Page 26, Figure 3.3}
        \PY{n}{x} \PY{o}{=} \PY{l+m+mi}{24}
        \PY{n}{epsilon} \PY{o}{=} \PY{l+m+mf}{0.01}
        \PY{n}{step} \PY{o}{=} \PY{n}{epsilon}\PY{o}{*}\PY{o}{*}\PY{l+m+mi}{2}
        \PY{n}{numGuesses} \PY{o}{=} \PY{l+m+mi}{0}
        \PY{n}{ans} \PY{o}{=} \PY{l+m+mf}{0.0}
        \PY{k}{while} \PY{n+nb}{abs}\PY{p}{(}\PY{n}{ans}\PY{o}{*}\PY{o}{*}\PY{l+m+mi}{2} \PY{o}{\PYZhy{}} \PY{n}{x}\PY{p}{)} \PY{o}{\PYZgt{}}\PY{o}{=} \PY{n}{epsilon} \PY{o+ow}{and} \PY{n}{ans} \PY{o}{\PYZlt{}}\PY{o}{=} \PY{n}{x}\PY{p}{:}
            \PY{n}{ans} \PY{o}{+}\PY{o}{=} \PY{n}{step}        \PY{c+c1}{\PYZsh{} += :ans = ans+ step; \PYZhy{}= *=                         }
            \PY{n}{numGuesses} \PY{o}{+}\PY{o}{=} \PY{l+m+mi}{1}
        \PY{n+nb}{print}\PY{p}{(}\PY{l+s+s1}{\PYZsq{}}\PY{l+s+s1}{numGuesses =}\PY{l+s+s1}{\PYZsq{}}\PY{p}{,} \PY{n}{numGuesses}\PY{p}{)}
        \PY{k}{if} \PY{n+nb}{abs}\PY{p}{(}\PY{n}{ans}\PY{o}{*}\PY{o}{*}\PY{l+m+mi}{2} \PY{o}{\PYZhy{}} \PY{n}{x}\PY{p}{)} \PY{o}{\PYZgt{}}\PY{o}{=} \PY{n}{epsilon}\PY{p}{:}
            \PY{n+nb}{print}\PY{p}{(}\PY{l+s+s1}{\PYZsq{}}\PY{l+s+s1}{Failed on square root of}\PY{l+s+s1}{\PYZsq{}}\PY{p}{,} \PY{n}{x}\PY{p}{)}
        \PY{k}{else}\PY{p}{:}
            \PY{n+nb}{print}\PY{p}{(}\PY{n}{ans}\PY{p}{,} \PY{l+s+s1}{\PYZsq{}}\PY{l+s+s1}{is close to square root of}\PY{l+s+s1}{\PYZsq{}}\PY{p}{,} \PY{n}{x}\PY{p}{)}
\end{Verbatim}


    \begin{Verbatim}[commandchars=\\\{\}]
{\color{incolor}In [{\color{incolor} }]:} \PY{n}{x}\PY{o}{=}\PY{l+m+mf}{0.25}  \PY{c+c1}{\PYZsh{}  0.5 [0,1]}
        \PY{n}{epsilon} \PY{o}{=} \PY{l+m+mf}{0.01}
        \PY{n}{step} \PY{o}{=} \PY{n}{epsilon}\PY{o}{*}\PY{o}{*}\PY{l+m+mi}{2}
        \PY{n}{numGuesses} \PY{o}{=} \PY{l+m+mi}{0}
        \PY{n}{ans} \PY{o}{=} \PY{l+m+mf}{0.0}
        \PY{k}{while} \PY{n+nb}{abs}\PY{p}{(}\PY{n}{ans}\PY{o}{*}\PY{o}{*}\PY{l+m+mi}{2} \PY{o}{\PYZhy{}} \PY{n}{x}\PY{p}{)} \PY{o}{\PYZgt{}}\PY{o}{=} \PY{n}{epsilon} \PY{o+ow}{and} \PY{n}{ans} \PY{o}{\PYZlt{}}\PY{o}{=} \PY{n}{x}\PY{p}{:}
            \PY{n}{ans} \PY{o}{+}\PY{o}{=} \PY{n}{step}        \PY{c+c1}{\PYZsh{} += :ans = ans+ step; \PYZhy{}= *=                         }
            \PY{n}{numGuesses} \PY{o}{+}\PY{o}{=} \PY{l+m+mi}{1}
        \PY{n+nb}{print}\PY{p}{(}\PY{l+s+s1}{\PYZsq{}}\PY{l+s+s1}{numGuesses =}\PY{l+s+s1}{\PYZsq{}}\PY{p}{,} \PY{n}{numGuesses}\PY{p}{)}
        \PY{k}{if} \PY{n+nb}{abs}\PY{p}{(}\PY{n}{ans}\PY{o}{*}\PY{o}{*}\PY{l+m+mi}{2} \PY{o}{\PYZhy{}} \PY{n}{x}\PY{p}{)} \PY{o}{\PYZgt{}}\PY{o}{=} \PY{n}{epsilon}\PY{p}{:}
            \PY{n+nb}{print}\PY{p}{(}\PY{l+s+s1}{\PYZsq{}}\PY{l+s+s1}{Failed on square root of}\PY{l+s+s1}{\PYZsq{}}\PY{p}{,} \PY{n}{x}\PY{p}{)}
        \PY{k}{else}\PY{p}{:}
            \PY{n+nb}{print}\PY{p}{(}\PY{n}{ans}\PY{p}{,} \PY{l+s+s1}{\PYZsq{}}\PY{l+s+s1}{is close to square root of}\PY{l+s+s1}{\PYZsq{}}\PY{p}{,} \PY{n}{x}\PY{p}{)}
\end{Verbatim}


    Exhaustive enumeration is a search technique that works only if the set
of values being searched includes the answer

    \begin{Verbatim}[commandchars=\\\{\}]
{\color{incolor}In [{\color{incolor} }]:} \PY{n}{x}\PY{o}{=}\PY{l+m+mf}{0.25}  \PY{c+c1}{\PYZsh{}  0.5 [0,1]}
        \PY{n}{epsilon} \PY{o}{=} \PY{l+m+mf}{0.01}
        \PY{n}{step} \PY{o}{=} \PY{n}{epsilon}\PY{o}{*}\PY{o}{*}\PY{l+m+mi}{2}  \PY{c+c1}{\PYZsh{}  epsilon**3 }
        \PY{n}{numGuesses} \PY{o}{=} \PY{l+m+mi}{0}  \PY{c+c1}{\PYZsh{} 3513631}
        \PY{n}{ans} \PY{o}{=} \PY{l+m+mf}{0.0}
        \PY{k}{while} \PY{n+nb}{abs}\PY{p}{(}\PY{n}{ans}\PY{o}{*}\PY{o}{*}\PY{l+m+mi}{2} \PY{o}{\PYZhy{}} \PY{n}{x}\PY{p}{)} \PY{o}{\PYZgt{}}\PY{o}{=} \PY{n}{epsilon} \PY{o+ow}{and} \PY{n}{ans}\PY{o}{*}\PY{n}{ans} \PY{o}{\PYZlt{}}\PY{o}{=} \PY{n}{x}\PY{p}{:}  \PY{c+c1}{\PYZsh{} }
            \PY{n}{ans} \PY{o}{+}\PY{o}{=} \PY{n}{step}        \PY{c+c1}{\PYZsh{} += :ans = ans+ step; \PYZhy{}= *=                         }
            \PY{n}{numGuesses} \PY{o}{+}\PY{o}{=} \PY{l+m+mi}{1}
        \PY{n+nb}{print}\PY{p}{(}\PY{l+s+s1}{\PYZsq{}}\PY{l+s+s1}{numGuesses =}\PY{l+s+s1}{\PYZsq{}}\PY{p}{,} \PY{n}{numGuesses}\PY{p}{)}
        \PY{k}{if} \PY{n+nb}{abs}\PY{p}{(}\PY{n}{ans}\PY{o}{*}\PY{o}{*}\PY{l+m+mi}{2} \PY{o}{\PYZhy{}} \PY{n}{x}\PY{p}{)} \PY{o}{\PYZgt{}}\PY{o}{=} \PY{n}{epsilon}\PY{p}{:}
            \PY{n+nb}{print}\PY{p}{(}\PY{l+s+s1}{\PYZsq{}}\PY{l+s+s1}{Failed on square root of}\PY{l+s+s1}{\PYZsq{}}\PY{p}{,} \PY{n}{x}\PY{p}{)}
        \PY{k}{else}\PY{p}{:}
            \PY{n+nb}{print}\PY{p}{(}\PY{n}{ans}\PY{p}{,} \PY{l+s+s1}{\PYZsq{}}\PY{l+s+s1}{is close to square root of}\PY{l+s+s1}{\PYZsq{}}\PY{p}{,} \PY{n}{x}\PY{p}{)}
\end{Verbatim}


    The time has come to look for a different way to attack the problem. We
need to choose a better algorithm rather than fine tune the current one.

    \hypertarget{using-bisection-search-to-approximate-square-root}{%
\subsubsection{2) Using bisection search to approximate square
root}\label{using-bisection-search-to-approximate-square-root}}

Suppose we know that a good approximation to the square root of x lies
somewhere between 0 and max. We can exploit the fact that numbers are
totally ordered.

Since we don't necessarily know where to start searching, let's start in
the middle.

\begin{verbatim}
0__________________________guess__________________________max
\end{verbatim}

    \begin{Verbatim}[commandchars=\\\{\}]
{\color{incolor}In [{\color{incolor} }]:} \PY{c+c1}{\PYZsh{} Page 28, Figure 3.4}
        \PY{c+c1}{\PYZsh{}x = 25}
        
        \PY{n}{x}\PY{o}{=}\PY{l+m+mi}{123456789}
        \PY{n}{epsilon} \PY{o}{=} \PY{l+m+mf}{0.01}
        \PY{n}{numGuesses} \PY{o}{=} \PY{l+m+mi}{0}
        \PY{n}{low} \PY{o}{=} \PY{l+m+mf}{0.0}
        \PY{n}{high} \PY{o}{=} \PY{n+nb}{max}\PY{p}{(}\PY{l+m+mf}{1.0}\PY{p}{,} \PY{n}{x}\PY{p}{)}        \PY{c+c1}{\PYZsh{} the square root of x lies somewhere between 0 and max}
        \PY{n}{ans} \PY{o}{=} \PY{p}{(}\PY{n}{high} \PY{o}{+} \PY{n}{low}\PY{p}{)} \PY{o}{/} \PY{l+m+mf}{2.0}  \PY{c+c1}{\PYZsh{} let’s start in the middle.}
        
        \PY{k}{while} \PY{n+nb}{abs}\PY{p}{(}\PY{n}{ans}\PY{o}{*}\PY{o}{*}\PY{l+m+mi}{2} \PY{o}{\PYZhy{}} \PY{n}{x}\PY{p}{)} \PY{o}{\PYZgt{}}\PY{o}{=} \PY{n}{epsilon}\PY{p}{:}
        \PY{c+c1}{\PYZsh{}    print(\PYZsq{}low =\PYZsq{}, low, \PYZsq{}high =\PYZsq{}, high, \PYZsq{}ans =\PYZsq{}, ans)}
         
        \PY{c+c1}{\PYZsh{}    print(\PYZsq{}root interval=[ \PYZpc{}12.9f\PYZsq{}\PYZpc{}low,\PYZsq{}, \PYZpc{}12.9f\PYZsq{}\PYZpc{}high,\PYZsq{}]\PYZsq{}, \PYZsq{}ans =\PYZpc{}12.9f\PYZsq{}\PYZpc{}(ans)) \PYZsh{}old string formatting}
            
            \PY{n+nb}{print}\PY{p}{(}\PY{l+s+s1}{\PYZsq{}}\PY{l+s+s1}{root interval=[ }\PY{l+s+si}{\PYZob{}0:.9f\PYZcb{}}\PY{l+s+s1}{, }\PY{l+s+si}{\PYZob{}0:.9f\PYZcb{}}\PY{l+s+s1}{], ans= }\PY{l+s+si}{\PYZob{}0:.9f\PYZcb{}}\PY{l+s+s1}{\PYZsq{}}\PY{o}{.}\PY{n}{format}\PY{p}{(}\PY{n}{low}\PY{p}{,}\PY{n}{high}\PY{p}{,}\PY{n}{ans}\PY{p}{)}\PY{p}{)}
           
            \PY{n}{numGuesses} \PY{o}{+}\PY{o}{=} \PY{l+m+mi}{1}
            
            \PY{c+c1}{\PYZsh{} whether it is too big or too small}
            \PY{k}{if} \PY{n}{ans}\PY{o}{*}\PY{o}{*}\PY{l+m+mi}{2} \PY{o}{\PYZlt{}} \PY{n}{x}\PY{p}{:}    
                \PY{n}{low} \PY{o}{=} \PY{n}{ans}    \PY{c+c1}{\PYZsh{} If it is too small, we know that the answer must lie to the right}
                             \PY{c+c1}{\PYZsh{} [0,max]\PYZhy{}\PYZgt{}[ans,max]}
            \PY{k}{else}\PY{p}{:}
                \PY{n}{high} \PY{o}{=} \PY{n}{ans}   \PY{c+c1}{\PYZsh{} If it is too big, we know that the answer must lie to the left.}
                             \PY{c+c1}{\PYZsh{} [0,max]\PYZhy{}[0,ans]}
            \PY{n}{ans} \PY{o}{=} \PY{p}{(}\PY{n}{high} \PY{o}{+} \PY{n}{low}\PY{p}{)} \PY{o}{/} \PY{l+m+mf}{2.0}
        
        \PY{n+nb}{print}\PY{p}{(}\PY{l+s+s1}{\PYZsq{}}\PY{l+s+se}{\PYZbs{}n}\PY{l+s+s1}{numGuesses =}\PY{l+s+s1}{\PYZsq{}}\PY{p}{,} \PY{n}{numGuesses}\PY{p}{)}
        \PY{n+nb}{print}\PY{p}{(}\PY{n}{ans}\PY{p}{,} \PY{l+s+s1}{\PYZsq{}}\PY{l+s+s1}{is close to square root of}\PY{l+s+s1}{\PYZsq{}}\PY{p}{,} \PY{n}{x}\PY{p}{)}
\end{Verbatim}


    Because it divides the search space in half at each step, it is called a
bisection search

    \hypertarget{python-tutorial-7.1-fancier-output-formatting}{%
\paragraph{Python Tutorial : 7.1 Fancier Output
Formatting}\label{python-tutorial-7.1-fancier-output-formatting}}

https://docs.python.org/3.5/tutorial/inputoutput.html\#fancier-output-formatting

    \hypertarget{a-few-words-about-using-floats}{%
\subsection{3.4 A Few Words About Using
Floats}\label{a-few-words-about-using-floats}}

    \begin{Verbatim}[commandchars=\\\{\}]
{\color{incolor}In [{\color{incolor} }]:} \PY{n}{x} \PY{o}{=} \PY{l+m+mf}{0.0}
        \PY{k}{for} \PY{n}{i} \PY{o+ow}{in} \PY{n+nb}{range}\PY{p}{(}\PY{l+m+mi}{10}\PY{p}{)}\PY{p}{:}
            \PY{n}{x} \PY{o}{=} \PY{n}{x} \PY{o}{+} \PY{l+m+mf}{0.1}    \PY{c+c1}{\PYZsh{} because the value to which x is bound is not exactly 1.0}
        
        \PY{k}{if} \PY{n}{x} \PY{o}{==} \PY{l+m+mf}{1.0}\PY{p}{:}
            \PY{n+nb}{print}\PY{p}{(}\PY{n}{x}\PY{p}{,} \PY{l+s+s1}{\PYZsq{}}\PY{l+s+s1}{= 1.0}\PY{l+s+s1}{\PYZsq{}}\PY{p}{)}
        \PY{k}{else}\PY{p}{:}
            \PY{n+nb}{print}\PY{p}{(}\PY{n}{x}\PY{p}{,} \PY{l+s+s1}{\PYZsq{}}\PY{l+s+s1}{is not 1.0}\PY{l+s+s1}{\PYZsq{}}\PY{p}{)}
            \PY{n+nb}{print}\PY{p}{(}\PY{l+s+s1}{\PYZsq{}}\PY{l+s+se}{\PYZbs{}n}\PY{l+s+s1}{ Tested that two floating point values are equal (==) instead of nearly equal}\PY{l+s+s1}{\PYZsq{}}\PY{p}{)}
        
        \PY{n+nb}{print}\PY{p}{(}\PY{l+s+s1}{\PYZsq{}}\PY{l+s+se}{\PYZbs{}n}\PY{l+s+s1}{ x==1.0 }\PY{l+s+s1}{\PYZsq{}}\PY{p}{,} \PY{n}{x}\PY{o}{==}\PY{l+m+mf}{1.0}\PY{p}{)}    
        \PY{n+nb}{print}\PY{p}{(}\PY{n}{x}\PY{p}{)}
\end{Verbatim}


    Why does it get to the else clause in the first place?

Modern computers use binary, not decimal, representations. We represent
the significant digits and exponents in binary rather than decimal and
raise 2 rather than 10 to the exponent.

\(sig*2^{exp}\)

0.625

\((101,-011) \rightarrow 5*2^{-3} \rightarrow 0.625\)

\$ 1/10-\textgreater{}0.1 ? sig*2\^{}\{exp\}\$

** if sig=1,exp=-3 **

\$ (01, -11) \rightarrow 1*2\^{}\{-3\} \rightarrow  1/8 = 0.125 \neq 0.1
\$

\$(0011, -0100) \rightarrow 3*2\^{}\{-4\} \rightarrow  3/32
\rightarrow 0.09375 \neq 0.1 \$

\$(11001, -01000) \rightarrow 25*2\^{}\{-8\} \rightarrow25/256
\rightarrow 0.09765625 \neq 0.1 \$

How many significant digits would we need to get an exact floating point
representation of 0.1? An infinite number of digits!

There do not exist integers sig and exp such that \(sig * 2^{-exp}\)
equals 0.1.

In base 2, 1/10 is the infinitely repeating fraction

0.0001100110011001100110011001100110011001100110011\ldots{}

In most Python implementations, there are 53 bits of precision available
for floating point numbers,

    \begin{Verbatim}[commandchars=\\\{\}]
{\color{incolor}In [{\color{incolor} }]:} \PY{n}{x}\PY{o}{=}\PY{l+m+mi}{1}\PY{o}{/}\PY{l+m+mi}{10}
        \PY{n+nb}{print}\PY{p}{(}\PY{n}{x}\PY{p}{)}  \PY{c+c1}{\PYZsh{} automatic rounding}
\end{Verbatim}


    Just remember, even though the printed result looks like the exact value
of 1/10, the actual stored value is the nearest representable binary
fraction.

if you want to explicitly round a floating point number, use the round
function. The expression

\begin{Shaded}
\begin{Highlighting}[]
\BuiltInTok{round}\NormalTok{(x, numDigits)}
\end{Highlighting}
\end{Shaded}

returns the floating point number equivalent to rounding the value of x
to numDigits decimal digits following the decimal point.

    \begin{Verbatim}[commandchars=\\\{\}]
{\color{incolor}In [{\color{incolor} }]:} \PY{n+nb}{round}\PY{p}{(}\PY{l+m+mi}{2}\PY{o}{*}\PY{o}{*}\PY{l+m+mf}{0.5}\PY{p}{,} \PY{l+m+mi}{4}\PY{p}{)}
\end{Verbatim}


    Does the difference between real and floating point numbers really
matter? Most of the time, mercifully, it does not.

However,

tests for equality write abs(x-y) \textless{} 0.0001 rather than x == y.

the accumulation of rounding errors

    \hypertarget{numpy.finfo}{%
\paragraph{numpy.finfo}\label{numpy.finfo}}

class numpy.finfo

\begin{verbatim}
Machine limits for floating point types.
\end{verbatim}

http://docs.scipy.org/doc/numpy/reference/generated/numpy.finfo.html

    \begin{Verbatim}[commandchars=\\\{\}]
{\color{incolor}In [{\color{incolor} }]:} \PY{k+kn}{import} \PY{n+nn}{numpy} \PY{k}{as} \PY{n+nn}{np}
        
        \PY{n}{iexp32}  \PY{o}{=} \PY{n}{np}\PY{o}{.}\PY{n}{finfo}\PY{p}{(}\PY{n}{np}\PY{o}{.}\PY{n}{float32}\PY{p}{)}\PY{o}{.}\PY{n}{iexp} 
        \PY{n+nb}{print}\PY{p}{(}\PY{l+s+s1}{\PYZsq{}}\PY{l+s+s1}{The number of bits in the exponent portion: }\PY{l+s+s1}{\PYZsq{}}\PY{p}{,}\PY{n}{iexp32}\PY{p}{)}
        \PY{c+c1}{\PYZsh{} nmant (int) The number of bits in the mantissa. }
        \PY{n}{nmant32} \PY{o}{=} \PY{n}{np}\PY{o}{.}\PY{n}{finfo}\PY{p}{(}\PY{l+s+s2}{\PYZdq{}}\PY{l+s+s2}{float32}\PY{l+s+s2}{\PYZdq{}}\PY{p}{)}\PY{o}{.}\PY{n}{nmant} 
        \PY{n+nb}{print}\PY{p}{(}\PY{l+s+s1}{\PYZsq{}}\PY{l+s+s1}{The number of bits in the mantissa: }\PY{l+s+s1}{\PYZsq{}}\PY{p}{,}\PY{n}{nmant32}\PY{p}{)}
        \PY{n}{eps32} \PY{o}{=} \PY{n}{np}\PY{o}{.}\PY{n}{finfo}\PY{p}{(}\PY{l+s+s2}{\PYZdq{}}\PY{l+s+s2}{float32}\PY{l+s+s2}{\PYZdq{}}\PY{p}{)}\PY{o}{.}\PY{n}{eps}
        \PY{n+nb}{print}\PY{p}{(}\PY{n}{eps32}\PY{p}{)}
\end{Verbatim}


    \begin{Verbatim}[commandchars=\\\{\}]
{\color{incolor}In [{\color{incolor} }]:} \PY{k}{for} \PY{n}{f} \PY{o+ow}{in} \PY{p}{(}\PY{n}{np}\PY{o}{.}\PY{n}{float16}\PY{p}{,}\PY{n}{np}\PY{o}{.}\PY{n}{float32}\PY{p}{,} \PY{n+nb}{float}\PY{p}{)}\PY{p}{:}
            \PY{n}{finfo} \PY{o}{=} \PY{n}{np}\PY{o}{.}\PY{n}{finfo}\PY{p}{(}\PY{n}{f}\PY{p}{)}
            \PY{n+nb}{print}\PY{p}{(}\PY{n}{finfo}\PY{p}{)}
\end{Verbatim}


    \hypertarget{further-reading}{%
\subsubsection{Further Reading}\label{further-reading}}

\hypertarget{python-tutorial-chapter-15-floating-point-arithmetic-issues-and-limitations}{%
\paragraph{1 Python Tutorial: Chapter 15 FLOATING POINT ARITHMETIC:
ISSUES AND
LIMITATIONS}\label{python-tutorial-chapter-15-floating-point-arithmetic-issues-and-limitations}}

For use cases which require exact decimal representation,

try using the decimal module which implements decimal arithmetic
suitable for accounting applications and high-precision applications.

Another form of exact arithmetic is supported by the fractions module
which implements arithmetic based on rational numbers (so the numbers
like 1/3 can be represented exactly).

\hypertarget{numerical-recipes-2-1.3-error-accuracy-and-stability}{%
\subparagraph{2 Numerical Recipes :2 1.3 Error, Accuracy, and
Stability}\label{numerical-recipes-2-1.3-error-accuracy-and-stability}}

http://numerical.recipes/

In floating-point representation, a number is represented internally by

\begin{enumerate}
\def\labelenumi{\arabic{enumi})}
\item
  a sign bit s : interpreted as plus or minus,
\item
  an exact integer exponent e,
\item
  an exact positive integer mantissa M.
\end{enumerate}

Taken together these represent the number

\(s*M* B^{e−E}\)

where B is the base of the representation (usually B = 2, but sometimes
B = 16),and E is the bias of the exponent, a fixed integer constant for
any given machine and representation.

An example is shown in Figure(Floating point representations of numbers
in a typical 32-bit (4-byte) format,Exponent bias E=127)

\begin{figure}
\centering
\includegraphics{./img/float.jpg}
\caption{float}
\end{figure}

\hypertarget{ieee-floating-point}{%
\paragraph{3 IEEE floating point}\label{ieee-floating-point}}

https://en.wikipedia.org/wiki/IEEE\_floating\_point

    \hypertarget{newton-raphson}{%
\subsection{3.5 Newton-Raphson}\label{newton-raphson}}

we shall look at it only in the context of finding the real roots of a
polynomial with one variable.

\(f(x)=a*x^n +a_0\)

Want to find r:

\(f(r)=0\)

Newton proved a theorem that implies that if a value, call it \(X_k\),
is an approximation to a root of a polynomial, then

\(x_{k+1}=x_k– \frac{f(x_k)}{f’(x_k)}\)

is a better approximation. where \(f’\) is the first derivative of
\(f\),

\(y=f(x_k)+f'(x_k)(x-x_k)\)

liner equation between two point: \((x_{k+1},0),(x_k,y_k)\)

\begin{figure}
\centering
\includegraphics{./img/newton.jpg}
\caption{newton}
\end{figure}

For example, the first derivative of

\(x^2 – k\) is \(2x\).

Therefore, we know that we can improve on the current guess, call it
\(x_k\)

by choosing as our next guess \(x_{k+1}\):

\(x_{k+1}=x_k - \frac{x_{k}^2 - k}{2x_k}\)

This is called successive approximation.

    \begin{Verbatim}[commandchars=\\\{\}]
{\color{incolor}In [{\color{incolor} }]:} \PY{c+c1}{\PYZsh{} Newton\PYZhy{}Raphson for square root}
        
        \PY{c+c1}{\PYZsh{}Find x such that x**2 \PYZhy{} 24 is within epsilon }
        
        \PY{n}{epsilon} \PY{o}{=} \PY{l+m+mf}{0.01}   \PY{c+c1}{\PYZsh{} 试验提示1:改变精度,测试死循环,给出改进的稳健 算法}
        
        \PY{n}{k} \PY{o}{=} \PY{l+m+mf}{24.0}
        
        \PY{n}{guess} \PY{o}{=}\PY{n}{k}\PY{o}{/}\PY{l+m+mf}{2.0}   \PY{c+c1}{\PYZsh{} reinitialize a variable,试验提示2: 比较不同初值下速度,如 0.01}
        
        \PY{c+c1}{\PYZsh{} 6.2.3 Failed to reinitialize a variable }
        
        \PY{n}{successiveapproximation}\PY{o}{=}\PY{l+m+mi}{0}  \PY{c+c1}{\PYZsh{}  试验提示3:和二分法对比下数值求解的 速度和精度}
        \PY{k}{while} \PY{n+nb}{abs}\PY{p}{(}\PY{n}{guess}\PY{o}{*}\PY{n}{guess} \PY{o}{\PYZhy{}} \PY{n}{k}\PY{p}{)} \PY{o}{\PYZgt{}}\PY{o}{=} \PY{n}{epsilon}\PY{p}{:}
            
            \PY{n}{guess} \PY{o}{=} \PY{n}{guess} \PY{o}{\PYZhy{}} \PY{p}{(}\PY{p}{(}\PY{p}{(}\PY{n}{guess}\PY{o}{*}\PY{o}{*}\PY{l+m+mi}{2}\PY{p}{)} \PY{o}{\PYZhy{}} \PY{n}{k}\PY{p}{)}\PY{o}{/}\PY{p}{(}\PY{l+m+mi}{2}\PY{o}{*}\PY{n}{guess}\PY{p}{)}\PY{p}{)}  \PY{c+c1}{\PYZsh{} a better next approximation}
            
            \PY{n}{successiveapproximation}\PY{o}{+}\PY{o}{=}\PY{l+m+mi}{1}
        
        \PY{n+nb}{print}\PY{p}{(}\PY{l+s+s1}{\PYZsq{}}\PY{l+s+s1}{Square root of}\PY{l+s+s1}{\PYZsq{}}\PY{p}{,} \PY{n}{k}\PY{p}{,} \PY{l+s+s1}{\PYZsq{}}\PY{l+s+s1}{is about}\PY{l+s+s1}{\PYZsq{}}\PY{p}{,} \PY{n}{guess}\PY{p}{)}
        \PY{n+nb}{print}\PY{p}{(}\PY{l+s+s1}{\PYZsq{}}\PY{l+s+s1}{counts of successive approximation=}\PY{l+s+s1}{\PYZsq{}}\PY{p}{,}\PY{n}{successiveapproximation}\PY{p}{)}
\end{Verbatim}


    \hypertarget{further-reading}{%
\subsubsection{Further Reading:}\label{further-reading}}

Numerical Recipes in C : http://numerical.recipes/

\begin{itemize}
\item
  9.1 Bracketing and Bisection
\item
  9.4 Newton-Raphson Method Using Derivative
\end{itemize}


    % Add a bibliography block to the postdoc
    
    
    
    \end{document}
