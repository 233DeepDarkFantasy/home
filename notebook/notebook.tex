
% Default to the notebook output style

    


% Inherit from the specified cell style.




    
\documentclass[11pt]{article}

    
    
    \usepackage[T1]{fontenc}
    % Nicer default font (+ math font) than Computer Modern for most use cases
    \usepackage{mathpazo}

    % Basic figure setup, for now with no caption control since it's done
    % automatically by Pandoc (which extracts ![](path) syntax from Markdown).
    \usepackage{graphicx}
    % We will generate all images so they have a width \maxwidth. This means
    % that they will get their normal width if they fit onto the page, but
    % are scaled down if they would overflow the margins.
    \makeatletter
    \def\maxwidth{\ifdim\Gin@nat@width>\linewidth\linewidth
    \else\Gin@nat@width\fi}
    \makeatother
    \let\Oldincludegraphics\includegraphics
    % Set max figure width to be 80% of text width, for now hardcoded.
    \renewcommand{\includegraphics}[1]{\Oldincludegraphics[width=.8\maxwidth]{#1}}
    % Ensure that by default, figures have no caption (until we provide a
    % proper Figure object with a Caption API and a way to capture that
    % in the conversion process - todo).
    \usepackage{caption}
    \DeclareCaptionLabelFormat{nolabel}{}
    \captionsetup{labelformat=nolabel}

    \usepackage{adjustbox} % Used to constrain images to a maximum size 
    \usepackage{xcolor} % Allow colors to be defined
    \usepackage{enumerate} % Needed for markdown enumerations to work
    \usepackage{geometry} % Used to adjust the document margins
    \usepackage{amsmath} % Equations
    \usepackage{amssymb} % Equations
    \usepackage{textcomp} % defines textquotesingle
    % Hack from http://tex.stackexchange.com/a/47451/13684:
    \AtBeginDocument{%
        \def\PYZsq{\textquotesingle}% Upright quotes in Pygmentized code
    }
    \usepackage{upquote} % Upright quotes for verbatim code
    \usepackage{eurosym} % defines \euro
    \usepackage[mathletters]{ucs} % Extended unicode (utf-8) support
    \usepackage[utf8x]{inputenc} % Allow utf-8 characters in the tex document
    \usepackage{fancyvrb} % verbatim replacement that allows latex
    \usepackage{grffile} % extends the file name processing of package graphics 
                         % to support a larger range 
    % The hyperref package gives us a pdf with properly built
    % internal navigation ('pdf bookmarks' for the table of contents,
    % internal cross-reference links, web links for URLs, etc.)
    \usepackage{hyperref}
    \usepackage{longtable} % longtable support required by pandoc >1.10
    \usepackage{booktabs}  % table support for pandoc > 1.12.2
    \usepackage[inline]{enumitem} % IRkernel/repr support (it uses the enumerate* environment)
    \usepackage[normalem]{ulem} % ulem is needed to support strikethroughs (\sout)
                                % normalem makes italics be italics, not underlines
    

    
    
    % Colors for the hyperref package
    \definecolor{urlcolor}{rgb}{0,.145,.698}
    \definecolor{linkcolor}{rgb}{.71,0.21,0.01}
    \definecolor{citecolor}{rgb}{.12,.54,.11}

    % ANSI colors
    \definecolor{ansi-black}{HTML}{3E424D}
    \definecolor{ansi-black-intense}{HTML}{282C36}
    \definecolor{ansi-red}{HTML}{E75C58}
    \definecolor{ansi-red-intense}{HTML}{B22B31}
    \definecolor{ansi-green}{HTML}{00A250}
    \definecolor{ansi-green-intense}{HTML}{007427}
    \definecolor{ansi-yellow}{HTML}{DDB62B}
    \definecolor{ansi-yellow-intense}{HTML}{B27D12}
    \definecolor{ansi-blue}{HTML}{208FFB}
    \definecolor{ansi-blue-intense}{HTML}{0065CA}
    \definecolor{ansi-magenta}{HTML}{D160C4}
    \definecolor{ansi-magenta-intense}{HTML}{A03196}
    \definecolor{ansi-cyan}{HTML}{60C6C8}
    \definecolor{ansi-cyan-intense}{HTML}{258F8F}
    \definecolor{ansi-white}{HTML}{C5C1B4}
    \definecolor{ansi-white-intense}{HTML}{A1A6B2}

    % commands and environments needed by pandoc snippets
    % extracted from the output of `pandoc -s`
    \providecommand{\tightlist}{%
      \setlength{\itemsep}{0pt}\setlength{\parskip}{0pt}}
    \DefineVerbatimEnvironment{Highlighting}{Verbatim}{commandchars=\\\{\}}
    % Add ',fontsize=\small' for more characters per line
    \newenvironment{Shaded}{}{}
    \newcommand{\KeywordTok}[1]{\textcolor[rgb]{0.00,0.44,0.13}{\textbf{{#1}}}}
    \newcommand{\DataTypeTok}[1]{\textcolor[rgb]{0.56,0.13,0.00}{{#1}}}
    \newcommand{\DecValTok}[1]{\textcolor[rgb]{0.25,0.63,0.44}{{#1}}}
    \newcommand{\BaseNTok}[1]{\textcolor[rgb]{0.25,0.63,0.44}{{#1}}}
    \newcommand{\FloatTok}[1]{\textcolor[rgb]{0.25,0.63,0.44}{{#1}}}
    \newcommand{\CharTok}[1]{\textcolor[rgb]{0.25,0.44,0.63}{{#1}}}
    \newcommand{\StringTok}[1]{\textcolor[rgb]{0.25,0.44,0.63}{{#1}}}
    \newcommand{\CommentTok}[1]{\textcolor[rgb]{0.38,0.63,0.69}{\textit{{#1}}}}
    \newcommand{\OtherTok}[1]{\textcolor[rgb]{0.00,0.44,0.13}{{#1}}}
    \newcommand{\AlertTok}[1]{\textcolor[rgb]{1.00,0.00,0.00}{\textbf{{#1}}}}
    \newcommand{\FunctionTok}[1]{\textcolor[rgb]{0.02,0.16,0.49}{{#1}}}
    \newcommand{\RegionMarkerTok}[1]{{#1}}
    \newcommand{\ErrorTok}[1]{\textcolor[rgb]{1.00,0.00,0.00}{\textbf{{#1}}}}
    \newcommand{\NormalTok}[1]{{#1}}
    
    % Additional commands for more recent versions of Pandoc
    \newcommand{\ConstantTok}[1]{\textcolor[rgb]{0.53,0.00,0.00}{{#1}}}
    \newcommand{\SpecialCharTok}[1]{\textcolor[rgb]{0.25,0.44,0.63}{{#1}}}
    \newcommand{\VerbatimStringTok}[1]{\textcolor[rgb]{0.25,0.44,0.63}{{#1}}}
    \newcommand{\SpecialStringTok}[1]{\textcolor[rgb]{0.73,0.40,0.53}{{#1}}}
    \newcommand{\ImportTok}[1]{{#1}}
    \newcommand{\DocumentationTok}[1]{\textcolor[rgb]{0.73,0.13,0.13}{\textit{{#1}}}}
    \newcommand{\AnnotationTok}[1]{\textcolor[rgb]{0.38,0.63,0.69}{\textbf{\textit{{#1}}}}}
    \newcommand{\CommentVarTok}[1]{\textcolor[rgb]{0.38,0.63,0.69}{\textbf{\textit{{#1}}}}}
    \newcommand{\VariableTok}[1]{\textcolor[rgb]{0.10,0.09,0.49}{{#1}}}
    \newcommand{\ControlFlowTok}[1]{\textcolor[rgb]{0.00,0.44,0.13}{\textbf{{#1}}}}
    \newcommand{\OperatorTok}[1]{\textcolor[rgb]{0.40,0.40,0.40}{{#1}}}
    \newcommand{\BuiltInTok}[1]{{#1}}
    \newcommand{\ExtensionTok}[1]{{#1}}
    \newcommand{\PreprocessorTok}[1]{\textcolor[rgb]{0.74,0.48,0.00}{{#1}}}
    \newcommand{\AttributeTok}[1]{\textcolor[rgb]{0.49,0.56,0.16}{{#1}}}
    \newcommand{\InformationTok}[1]{\textcolor[rgb]{0.38,0.63,0.69}{\textbf{\textit{{#1}}}}}
    \newcommand{\WarningTok}[1]{\textcolor[rgb]{0.38,0.63,0.69}{\textbf{\textit{{#1}}}}}
    
    
    % Define a nice break command that doesn't care if a line doesn't already
    % exist.
    \def\br{\hspace*{\fill} \\* }
    % Math Jax compatability definitions
    \def\gt{>}
    \def\lt{<}
    % Document parameters
    \title{02\_INTRODUCTION\_TO\_PYTHON}
    
    
    

    % Pygments definitions
    
\makeatletter
\def\PY@reset{\let\PY@it=\relax \let\PY@bf=\relax%
    \let\PY@ul=\relax \let\PY@tc=\relax%
    \let\PY@bc=\relax \let\PY@ff=\relax}
\def\PY@tok#1{\csname PY@tok@#1\endcsname}
\def\PY@toks#1+{\ifx\relax#1\empty\else%
    \PY@tok{#1}\expandafter\PY@toks\fi}
\def\PY@do#1{\PY@bc{\PY@tc{\PY@ul{%
    \PY@it{\PY@bf{\PY@ff{#1}}}}}}}
\def\PY#1#2{\PY@reset\PY@toks#1+\relax+\PY@do{#2}}

\expandafter\def\csname PY@tok@m\endcsname{\def\PY@tc##1{\textcolor[rgb]{0.40,0.40,0.40}{##1}}}
\expandafter\def\csname PY@tok@ss\endcsname{\def\PY@tc##1{\textcolor[rgb]{0.10,0.09,0.49}{##1}}}
\expandafter\def\csname PY@tok@gh\endcsname{\let\PY@bf=\textbf\def\PY@tc##1{\textcolor[rgb]{0.00,0.00,0.50}{##1}}}
\expandafter\def\csname PY@tok@nl\endcsname{\def\PY@tc##1{\textcolor[rgb]{0.63,0.63,0.00}{##1}}}
\expandafter\def\csname PY@tok@nt\endcsname{\let\PY@bf=\textbf\def\PY@tc##1{\textcolor[rgb]{0.00,0.50,0.00}{##1}}}
\expandafter\def\csname PY@tok@gr\endcsname{\def\PY@tc##1{\textcolor[rgb]{1.00,0.00,0.00}{##1}}}
\expandafter\def\csname PY@tok@gt\endcsname{\def\PY@tc##1{\textcolor[rgb]{0.00,0.27,0.87}{##1}}}
\expandafter\def\csname PY@tok@cp\endcsname{\def\PY@tc##1{\textcolor[rgb]{0.74,0.48,0.00}{##1}}}
\expandafter\def\csname PY@tok@ch\endcsname{\let\PY@it=\textit\def\PY@tc##1{\textcolor[rgb]{0.25,0.50,0.50}{##1}}}
\expandafter\def\csname PY@tok@nn\endcsname{\let\PY@bf=\textbf\def\PY@tc##1{\textcolor[rgb]{0.00,0.00,1.00}{##1}}}
\expandafter\def\csname PY@tok@o\endcsname{\def\PY@tc##1{\textcolor[rgb]{0.40,0.40,0.40}{##1}}}
\expandafter\def\csname PY@tok@s\endcsname{\def\PY@tc##1{\textcolor[rgb]{0.73,0.13,0.13}{##1}}}
\expandafter\def\csname PY@tok@go\endcsname{\def\PY@tc##1{\textcolor[rgb]{0.53,0.53,0.53}{##1}}}
\expandafter\def\csname PY@tok@mf\endcsname{\def\PY@tc##1{\textcolor[rgb]{0.40,0.40,0.40}{##1}}}
\expandafter\def\csname PY@tok@nb\endcsname{\def\PY@tc##1{\textcolor[rgb]{0.00,0.50,0.00}{##1}}}
\expandafter\def\csname PY@tok@gd\endcsname{\def\PY@tc##1{\textcolor[rgb]{0.63,0.00,0.00}{##1}}}
\expandafter\def\csname PY@tok@vi\endcsname{\def\PY@tc##1{\textcolor[rgb]{0.10,0.09,0.49}{##1}}}
\expandafter\def\csname PY@tok@mo\endcsname{\def\PY@tc##1{\textcolor[rgb]{0.40,0.40,0.40}{##1}}}
\expandafter\def\csname PY@tok@s1\endcsname{\def\PY@tc##1{\textcolor[rgb]{0.73,0.13,0.13}{##1}}}
\expandafter\def\csname PY@tok@gp\endcsname{\let\PY@bf=\textbf\def\PY@tc##1{\textcolor[rgb]{0.00,0.00,0.50}{##1}}}
\expandafter\def\csname PY@tok@dl\endcsname{\def\PY@tc##1{\textcolor[rgb]{0.73,0.13,0.13}{##1}}}
\expandafter\def\csname PY@tok@sd\endcsname{\let\PY@it=\textit\def\PY@tc##1{\textcolor[rgb]{0.73,0.13,0.13}{##1}}}
\expandafter\def\csname PY@tok@ni\endcsname{\let\PY@bf=\textbf\def\PY@tc##1{\textcolor[rgb]{0.60,0.60,0.60}{##1}}}
\expandafter\def\csname PY@tok@s2\endcsname{\def\PY@tc##1{\textcolor[rgb]{0.73,0.13,0.13}{##1}}}
\expandafter\def\csname PY@tok@nf\endcsname{\def\PY@tc##1{\textcolor[rgb]{0.00,0.00,1.00}{##1}}}
\expandafter\def\csname PY@tok@kt\endcsname{\def\PY@tc##1{\textcolor[rgb]{0.69,0.00,0.25}{##1}}}
\expandafter\def\csname PY@tok@vc\endcsname{\def\PY@tc##1{\textcolor[rgb]{0.10,0.09,0.49}{##1}}}
\expandafter\def\csname PY@tok@vg\endcsname{\def\PY@tc##1{\textcolor[rgb]{0.10,0.09,0.49}{##1}}}
\expandafter\def\csname PY@tok@no\endcsname{\def\PY@tc##1{\textcolor[rgb]{0.53,0.00,0.00}{##1}}}
\expandafter\def\csname PY@tok@w\endcsname{\def\PY@tc##1{\textcolor[rgb]{0.73,0.73,0.73}{##1}}}
\expandafter\def\csname PY@tok@si\endcsname{\let\PY@bf=\textbf\def\PY@tc##1{\textcolor[rgb]{0.73,0.40,0.53}{##1}}}
\expandafter\def\csname PY@tok@cs\endcsname{\let\PY@it=\textit\def\PY@tc##1{\textcolor[rgb]{0.25,0.50,0.50}{##1}}}
\expandafter\def\csname PY@tok@nc\endcsname{\let\PY@bf=\textbf\def\PY@tc##1{\textcolor[rgb]{0.00,0.00,1.00}{##1}}}
\expandafter\def\csname PY@tok@kc\endcsname{\let\PY@bf=\textbf\def\PY@tc##1{\textcolor[rgb]{0.00,0.50,0.00}{##1}}}
\expandafter\def\csname PY@tok@il\endcsname{\def\PY@tc##1{\textcolor[rgb]{0.40,0.40,0.40}{##1}}}
\expandafter\def\csname PY@tok@bp\endcsname{\def\PY@tc##1{\textcolor[rgb]{0.00,0.50,0.00}{##1}}}
\expandafter\def\csname PY@tok@mh\endcsname{\def\PY@tc##1{\textcolor[rgb]{0.40,0.40,0.40}{##1}}}
\expandafter\def\csname PY@tok@ne\endcsname{\let\PY@bf=\textbf\def\PY@tc##1{\textcolor[rgb]{0.82,0.25,0.23}{##1}}}
\expandafter\def\csname PY@tok@kd\endcsname{\let\PY@bf=\textbf\def\PY@tc##1{\textcolor[rgb]{0.00,0.50,0.00}{##1}}}
\expandafter\def\csname PY@tok@kn\endcsname{\let\PY@bf=\textbf\def\PY@tc##1{\textcolor[rgb]{0.00,0.50,0.00}{##1}}}
\expandafter\def\csname PY@tok@k\endcsname{\let\PY@bf=\textbf\def\PY@tc##1{\textcolor[rgb]{0.00,0.50,0.00}{##1}}}
\expandafter\def\csname PY@tok@c\endcsname{\let\PY@it=\textit\def\PY@tc##1{\textcolor[rgb]{0.25,0.50,0.50}{##1}}}
\expandafter\def\csname PY@tok@sc\endcsname{\def\PY@tc##1{\textcolor[rgb]{0.73,0.13,0.13}{##1}}}
\expandafter\def\csname PY@tok@c1\endcsname{\let\PY@it=\textit\def\PY@tc##1{\textcolor[rgb]{0.25,0.50,0.50}{##1}}}
\expandafter\def\csname PY@tok@kr\endcsname{\let\PY@bf=\textbf\def\PY@tc##1{\textcolor[rgb]{0.00,0.50,0.00}{##1}}}
\expandafter\def\csname PY@tok@nd\endcsname{\def\PY@tc##1{\textcolor[rgb]{0.67,0.13,1.00}{##1}}}
\expandafter\def\csname PY@tok@sa\endcsname{\def\PY@tc##1{\textcolor[rgb]{0.73,0.13,0.13}{##1}}}
\expandafter\def\csname PY@tok@ow\endcsname{\let\PY@bf=\textbf\def\PY@tc##1{\textcolor[rgb]{0.67,0.13,1.00}{##1}}}
\expandafter\def\csname PY@tok@gi\endcsname{\def\PY@tc##1{\textcolor[rgb]{0.00,0.63,0.00}{##1}}}
\expandafter\def\csname PY@tok@mi\endcsname{\def\PY@tc##1{\textcolor[rgb]{0.40,0.40,0.40}{##1}}}
\expandafter\def\csname PY@tok@ge\endcsname{\let\PY@it=\textit}
\expandafter\def\csname PY@tok@vm\endcsname{\def\PY@tc##1{\textcolor[rgb]{0.10,0.09,0.49}{##1}}}
\expandafter\def\csname PY@tok@sh\endcsname{\def\PY@tc##1{\textcolor[rgb]{0.73,0.13,0.13}{##1}}}
\expandafter\def\csname PY@tok@sr\endcsname{\def\PY@tc##1{\textcolor[rgb]{0.73,0.40,0.53}{##1}}}
\expandafter\def\csname PY@tok@cpf\endcsname{\let\PY@it=\textit\def\PY@tc##1{\textcolor[rgb]{0.25,0.50,0.50}{##1}}}
\expandafter\def\csname PY@tok@sb\endcsname{\def\PY@tc##1{\textcolor[rgb]{0.73,0.13,0.13}{##1}}}
\expandafter\def\csname PY@tok@gu\endcsname{\let\PY@bf=\textbf\def\PY@tc##1{\textcolor[rgb]{0.50,0.00,0.50}{##1}}}
\expandafter\def\csname PY@tok@gs\endcsname{\let\PY@bf=\textbf}
\expandafter\def\csname PY@tok@sx\endcsname{\def\PY@tc##1{\textcolor[rgb]{0.00,0.50,0.00}{##1}}}
\expandafter\def\csname PY@tok@na\endcsname{\def\PY@tc##1{\textcolor[rgb]{0.49,0.56,0.16}{##1}}}
\expandafter\def\csname PY@tok@fm\endcsname{\def\PY@tc##1{\textcolor[rgb]{0.00,0.00,1.00}{##1}}}
\expandafter\def\csname PY@tok@nv\endcsname{\def\PY@tc##1{\textcolor[rgb]{0.10,0.09,0.49}{##1}}}
\expandafter\def\csname PY@tok@se\endcsname{\let\PY@bf=\textbf\def\PY@tc##1{\textcolor[rgb]{0.73,0.40,0.13}{##1}}}
\expandafter\def\csname PY@tok@cm\endcsname{\let\PY@it=\textit\def\PY@tc##1{\textcolor[rgb]{0.25,0.50,0.50}{##1}}}
\expandafter\def\csname PY@tok@err\endcsname{\def\PY@bc##1{\setlength{\fboxsep}{0pt}\fcolorbox[rgb]{1.00,0.00,0.00}{1,1,1}{\strut ##1}}}
\expandafter\def\csname PY@tok@mb\endcsname{\def\PY@tc##1{\textcolor[rgb]{0.40,0.40,0.40}{##1}}}
\expandafter\def\csname PY@tok@kp\endcsname{\def\PY@tc##1{\textcolor[rgb]{0.00,0.50,0.00}{##1}}}

\def\PYZbs{\char`\\}
\def\PYZus{\char`\_}
\def\PYZob{\char`\{}
\def\PYZcb{\char`\}}
\def\PYZca{\char`\^}
\def\PYZam{\char`\&}
\def\PYZlt{\char`\<}
\def\PYZgt{\char`\>}
\def\PYZsh{\char`\#}
\def\PYZpc{\char`\%}
\def\PYZdl{\char`\$}
\def\PYZhy{\char`\-}
\def\PYZsq{\char`\'}
\def\PYZdq{\char`\"}
\def\PYZti{\char`\~}
% for compatibility with earlier versions
\def\PYZat{@}
\def\PYZlb{[}
\def\PYZrb{]}
\makeatother


    % Exact colors from NB
    \definecolor{incolor}{rgb}{0.0, 0.0, 0.5}
    \definecolor{outcolor}{rgb}{0.545, 0.0, 0.0}



    
    % Prevent overflowing lines due to hard-to-break entities
    \sloppy 
    % Setup hyperref package
    \hypersetup{
      breaklinks=true,  % so long urls are correctly broken across lines
      colorlinks=true,
      urlcolor=urlcolor,
      linkcolor=linkcolor,
      citecolor=citecolor,
      }
    % Slightly bigger margins than the latex defaults
    
    \geometry{verbose,tmargin=1in,bmargin=1in,lmargin=1in,rmargin=1in}
    
    

    \begin{document}
    
    
    \maketitle
    
    

    
    \hypertarget{introduction-to-python}{%
\section{2 Introduction to Python}\label{introduction-to-python}}

    \hypertarget{python-is-a-living-language.-since-its-introduction-by-guido-von-rossum-in-1990-it-has-undergone-many-changes.}{%
\paragraph{Python is a living language. Since its introduction by Guido
von Rossum in 1990, it has undergone many
changes.}\label{python-is-a-living-language.-since-its-introduction-by-guido-von-rossum-in-1990-it-has-undergone-many-changes.}}

For the first decade of its life, Python was a little known and little
used language. That changed with the arrival of Python 2.0 in 2000. In
addition to incorporating a number of important improvements to the
language itself, it marked a shift in the evolutionary path of the
language. A large number of people began developing libraries that
interfaced seamlessly with Python, and continuing support and
development of the Python ecosystem became a community-based activity.

Python 3.0 was released at the end of 2008. This version of Python
cleaned up many of the inconsistencies in the design of the various
releases of Python 2 (often referred to as Python 2.x). However, it was
not backward compatible. That meant that most programs written for
earlier versions of Python could not be run using implementations of
Python 3.0.

We use Python 3 throughout this course.

Officeal Web Site: https://www.python.org/

The mission of the Python Software Foundation is to promote, protect,
and advance the Python programming language, and to support and
facilitate the growth of a diverse and international community of Python
programmers.

Python is powerful\ldots{} and fast; plays well with others; runs
everywhere; is friendly \& easy to learn; is Open.

Life is short,You need Python

-- Bruce Eckel,ANSI C++ Comitee member

    \hypertarget{the-basic-elements-of-python}{%
\subsection{2.1 The Basic Elements of
Python}\label{the-basic-elements-of-python}}

    A Python program, sometimes called a script, is a sequence of
definitions and commands.

These definitions are evaluated and the commands are executed by the
Python interpreter in something called the shell.

Typically, a new shell is created whenever execution of a program
begins. In most cases, a window is associated with the shell.

The symbol \textgreater{}\textgreater{}\textgreater{} is a shell prompt
indicating that the interpreter is expecting the user to type some
Python code into the shell.

Python Shell in CMD

    \begin{Verbatim}[commandchars=\\\{\}]
{\color{incolor}In [{\color{incolor}2}]:} \PY{n+nb}{print}\PY{p}{(}\PY{l+s+s1}{\PYZsq{}}\PY{l+s+s1}{Yankees rule!}\PY{l+s+s1}{\PYZsq{}}\PY{p}{)}
        \PY{n+nb}{print}\PY{p}{(}\PY{l+s+s1}{\PYZsq{}}\PY{l+s+s1}{But not in Boston!}\PY{l+s+s1}{\PYZsq{}}\PY{p}{)}
        \PY{n+nb}{print}\PY{p}{(}\PY{l+s+s1}{\PYZsq{}}\PY{l+s+s1}{Yankees rule,}\PY{l+s+s1}{\PYZsq{}}\PY{p}{,} \PY{l+s+s1}{\PYZsq{}}\PY{l+s+s1}{but not in Boston!}\PY{l+s+s1}{\PYZsq{}}\PY{p}{)}
        \PY{n+nb}{print}\PY{p}{(}\PY{l+s+s1}{\PYZsq{}}\PY{l+s+s1}{SEU}\PY{l+s+s1}{\PYZsq{}}\PY{p}{)}
\end{Verbatim}


    \begin{Verbatim}[commandchars=\\\{\}]
Yankees rule!
But not in Boston!
Yankees rule, but not in Boston!
SEU

    \end{Verbatim}

    \hypertarget{objects-expressions-and-numerical-types}{%
\subsection{2.1.1 Objects, Expressions, and Numerical
Types}\label{objects-expressions-and-numerical-types}}

    \textbf{Objects} are the core things that Python programs manipulate.
Every object has a \textbf{type} that defines the kinds of things that
programs can do with objects of that type.

    Python has four basic types:

\begin{itemize}
\tightlist
\item
  \textbf{int} integers: -3 or 5 or 10002
\item
  \textbf{float} real numbers: 3.0 or 3.17 or -28.72,scientific
  notation: 1.6E3
\item
  \textbf{bool} the Boolean values True and False
\item
  None is a type with a single value.
\end{itemize}

    Objects and operators can be combined to form expressions, each of which
evaluates to an object of some type. We will refer to this as the value
of the expression.

    \begin{Verbatim}[commandchars=\\\{\}]
{\color{incolor}In [{\color{incolor} }]:} \PY{l+m+mi}{3} \PY{o}{+} \PY{l+m+mi}{2}
\end{Verbatim}


    \begin{Verbatim}[commandchars=\\\{\}]
{\color{incolor}In [{\color{incolor} }]:} \PY{l+m+mf}{3.0} \PY{o}{+} \PY{l+m+mf}{2.0}
\end{Verbatim}


    \begin{Verbatim}[commandchars=\\\{\}]
{\color{incolor}In [{\color{incolor}8}]:} \PY{l+m+mi}{3} \PY{o}{!=} \PY{l+m+mi}{3}
\end{Verbatim}


\begin{Verbatim}[commandchars=\\\{\}]
{\color{outcolor}Out[{\color{outcolor}8}]:} False
\end{Verbatim}
            
    The built-in Python function type can be used to find out the type of an
object:

    \begin{Verbatim}[commandchars=\\\{\}]
{\color{incolor}In [{\color{incolor}9}]:} \PY{n+nb}{type}\PY{p}{(}\PY{l+m+mi}{3}\PY{p}{)}
\end{Verbatim}


\begin{Verbatim}[commandchars=\\\{\}]
{\color{outcolor}Out[{\color{outcolor}9}]:} int
\end{Verbatim}
            
    \begin{Verbatim}[commandchars=\\\{\}]
{\color{incolor}In [{\color{incolor}10}]:} \PY{n+nb}{type}\PY{p}{(}\PY{l+m+mf}{3.0}\PY{p}{)}
\end{Verbatim}


\begin{Verbatim}[commandchars=\\\{\}]
{\color{outcolor}Out[{\color{outcolor}10}]:} float
\end{Verbatim}
            
    \begin{Verbatim}[commandchars=\\\{\}]
{\color{incolor}In [{\color{incolor}11}]:} \PY{n+nb}{type}\PY{p}{(}\PY{k+kc}{True}\PY{p}{)}
\end{Verbatim}


\begin{Verbatim}[commandchars=\\\{\}]
{\color{outcolor}Out[{\color{outcolor}11}]:} bool
\end{Verbatim}
            
    \begin{Verbatim}[commandchars=\\\{\}]
{\color{incolor}In [{\color{incolor}16}]:} \PY{n}{TRue}\PY{o}{=}\PY{k+kc}{None}
         \PY{n+nb}{print}\PY{p}{(}\PY{n+nb}{type}\PY{p}{(}\PY{n}{TRue}\PY{p}{)}\PY{p}{)} \PY{c+c1}{\PYZsh{} True Only!}
         
         \PY{n}{TRue}\PY{o}{=}\PY{k+kc}{True}
         \PY{n+nb}{print}\PY{p}{(}\PY{n+nb}{type}\PY{p}{(}\PY{n}{TRue}\PY{p}{)}\PY{p}{)}
         \PY{n}{TRue}\PY{o}{=}\PY{l+m+mi}{1}
         \PY{n+nb}{print}\PY{p}{(}\PY{n+nb}{type}\PY{p}{(}\PY{n}{TRue}\PY{p}{)}\PY{p}{)}
\end{Verbatim}


    \begin{Verbatim}[commandchars=\\\{\}]
<class 'NoneType'>
<class 'bool'>
<class 'int'>

    \end{Verbatim}

    \begin{Verbatim}[commandchars=\\\{\}]
{\color{incolor}In [{\color{incolor} }]:} \PY{n+nb}{type}\PY{p}{(}\PY{k+kc}{None}\PY{p}{)}
\end{Verbatim}


    \begin{Verbatim}[commandchars=\\\{\}]
{\color{incolor}In [{\color{incolor} }]:} \PY{n+nb}{type}\PY{p}{(}\PY{n}{NONE}\PY{p}{)} \PY{c+c1}{\PYZsh{} None Only!!}
\end{Verbatim}


    \textbf{Operators} on types int and float

\begin{Shaded}
\begin{Highlighting}[]
\OperatorTok{+} \OperatorTok{-} \OperatorTok{*} \OperatorTok{//} \OperatorTok{/} \OperatorTok{%} \OperatorTok{**}
\end{Highlighting}
\end{Shaded}

    \begin{Verbatim}[commandchars=\\\{\}]
{\color{incolor}In [{\color{incolor} }]:} \PY{l+m+mi}{2}\PY{o}{+}\PY{l+m+mi}{3} \PY{c+c1}{\PYZsh{}i+j is the sum of i and j. If i and j are both of type int, the result is an int.}
\end{Verbatim}


    \begin{Verbatim}[commandchars=\\\{\}]
{\color{incolor}In [{\color{incolor} }]:} \PY{l+m+mf}{2.0}\PY{o}{+}\PY{l+m+mi}{3} \PY{c+c1}{\PYZsh{}If either of them is a float, the result is a float}
\end{Verbatim}


    \begin{Verbatim}[commandchars=\\\{\}]
{\color{incolor}In [{\color{incolor} }]:} \PY{l+m+mi}{7}\PY{o}{\PYZhy{}}\PY{l+m+mi}{2} \PY{c+c1}{\PYZsh{} i–j is i minus j. If i and j are both of type int, the result is an int.}
\end{Verbatim}


    \begin{Verbatim}[commandchars=\\\{\}]
{\color{incolor}In [{\color{incolor} }]:} \PY{l+m+mf}{7.12}\PY{o}{\PYZhy{}}\PY{l+m+mi}{3} \PY{c+c1}{\PYZsh{} If either of them is a float, the result is a float.}
\end{Verbatim}


    \begin{Verbatim}[commandchars=\\\{\}]
{\color{incolor}In [{\color{incolor} }]:} \PY{l+m+mi}{3}\PY{o}{*}\PY{l+m+mi}{6} \PY{c+c1}{\PYZsh{} i*j is the product of i and j. If i and j are both of type int, the result is an int.}
\end{Verbatim}


    \begin{Verbatim}[commandchars=\\\{\}]
{\color{incolor}In [{\color{incolor} }]:} \PY{l+m+mf}{3.12}\PY{o}{*}\PY{l+m+mi}{6} \PY{c+c1}{\PYZsh{} If either of them is a float, the result is a float.}
\end{Verbatim}


    \begin{Verbatim}[commandchars=\\\{\}]
{\color{incolor}In [{\color{incolor}17}]:} \PY{l+m+mi}{6}\PY{o}{/}\PY{o}{/}\PY{l+m+mi}{2} \PY{c+c1}{\PYZsh{} integer division returns the quotient and ignores the remainder.}
\end{Verbatim}


\begin{Verbatim}[commandchars=\\\{\}]
{\color{outcolor}Out[{\color{outcolor}17}]:} 3
\end{Verbatim}
            
    \begin{Verbatim}[commandchars=\\\{\}]
{\color{incolor}In [{\color{incolor}18}]:} \PY{l+m+mi}{7}\PY{o}{/}\PY{o}{/}\PY{l+m+mi}{2}  \PY{c+c1}{\PYZsh{} integer division returns the quotient and ignores the remainder.}
\end{Verbatim}


\begin{Verbatim}[commandchars=\\\{\}]
{\color{outcolor}Out[{\color{outcolor}18}]:} 3
\end{Verbatim}
            
    \begin{Verbatim}[commandchars=\\\{\}]
{\color{incolor}In [{\color{incolor} }]:} \PY{l+m+mi}{6}\PY{o}{/}\PY{l+m+mi}{2} \PY{c+c1}{\PYZsh{}  In Python 3, the / operator  always returns a float}
\end{Verbatim}


    \begin{Verbatim}[commandchars=\\\{\}]
{\color{incolor}In [{\color{incolor} }]:} \PY{l+m+mi}{7}\PY{o}{/}\PY{l+m+mi}{2} \PY{c+c1}{\PYZsh{}  In Python 3, the / operator  always returns a float}
\end{Verbatim}


    \begin{Verbatim}[commandchars=\\\{\}]
{\color{incolor}In [{\color{incolor}19}]:} \PY{l+m+mi}{7}\PY{o}{\PYZpc{}}\PY{k}{2} \PYZsh{} remainder,i\PYZpc{}6 i mod j
\end{Verbatim}


\begin{Verbatim}[commandchars=\\\{\}]
{\color{outcolor}Out[{\color{outcolor}19}]:} 1
\end{Verbatim}
            
    \begin{Verbatim}[commandchars=\\\{\}]
{\color{incolor}In [{\color{incolor}20}]:} \PY{l+m+mi}{2}\PY{o}{*}\PY{o}{*}\PY{l+m+mi}{3} \PY{c+c1}{\PYZsh{}i**j is i raised to the power j. If i and j are both of type int, the result is an int.}
\end{Verbatim}


\begin{Verbatim}[commandchars=\\\{\}]
{\color{outcolor}Out[{\color{outcolor}20}]:} 8
\end{Verbatim}
            
    \begin{Verbatim}[commandchars=\\\{\}]
{\color{incolor}In [{\color{incolor}21}]:} \PY{l+m+mf}{2.1}\PY{o}{*}\PY{o}{*}\PY{l+m+mi}{3}   \PY{c+c1}{\PYZsh{} If either of them is a float, the result is a float}
\end{Verbatim}


\begin{Verbatim}[commandchars=\\\{\}]
{\color{outcolor}Out[{\color{outcolor}21}]:} 9.261000000000001
\end{Verbatim}
            
    \begin{Verbatim}[commandchars=\\\{\}]
{\color{incolor}In [{\color{incolor}22}]:} \PY{l+m+mi}{2}\PY{o}{*}\PY{o}{*}\PY{l+m+mf}{3.1}  \PY{c+c1}{\PYZsh{} If either of them is a float, the result is a float}
\end{Verbatim}


\begin{Verbatim}[commandchars=\\\{\}]
{\color{outcolor}Out[{\color{outcolor}22}]:} 8.574187700290345
\end{Verbatim}
            
    The \textbf{comparison operators} are:

\begin{Shaded}
\begin{Highlighting}[]
   \OperatorTok{==} \CommentTok{#equal}
   \OperatorTok{!=} \CommentTok{#not equal}
   \OperatorTok{>}  \CommentTok{#greater}
   \OperatorTok{>=} \CommentTok{#at least}
   \OperatorTok{<}  \CommentTok{#less}
   \OperatorTok{<=} \CommentTok{#at most}
\end{Highlighting}
\end{Shaded}

    \begin{Verbatim}[commandchars=\\\{\}]
{\color{incolor}In [{\color{incolor} }]:} \PY{l+m+mi}{1}\PY{o}{==}\PY{l+m+mi}{2}
\end{Verbatim}


    \begin{Verbatim}[commandchars=\\\{\}]
{\color{incolor}In [{\color{incolor} }]:} \PY{l+m+mi}{1}\PY{o}{\PYZlt{}}\PY{o}{=}\PY{l+m+mi}{2}
\end{Verbatim}


    The operators on type bool are:

\begin{Shaded}
\begin{Highlighting}[]
\NormalTok{  a }\KeywordTok{and}\NormalTok{ b   }\CommentTok{# True if both a and b are True, and False otherwise.}
\NormalTok{  a }\KeywordTok{or}\NormalTok{  b   }\CommentTok{# True if at least one of a or b is True, and False otherwise.}
  \KeywordTok{not}\NormalTok{   a   }\CommentTok{#True if a is False, and False if a is True.}
\end{Highlighting}
\end{Shaded}

    \begin{Verbatim}[commandchars=\\\{\}]
{\color{incolor}In [{\color{incolor}23}]:} \PY{l+m+mi}{2}\PY{o}{\PYZgt{}}\PY{l+m+mi}{1} \PY{o+ow}{and} \PY{l+m+mi}{3}\PY{o}{\PYZgt{}}\PY{l+m+mi}{2}
\end{Verbatim}


\begin{Verbatim}[commandchars=\\\{\}]
{\color{outcolor}Out[{\color{outcolor}23}]:} True
\end{Verbatim}
            
    \begin{Verbatim}[commandchars=\\\{\}]
{\color{incolor}In [{\color{incolor} }]:} \PY{l+m+mi}{1}\PY{o}{\PYZgt{}}\PY{l+m+mi}{2} \PY{o+ow}{or} \PY{l+m+mi}{3}\PY{o}{\PYZgt{}}\PY{l+m+mi}{2}
\end{Verbatim}


    \begin{Verbatim}[commandchars=\\\{\}]
{\color{incolor}In [{\color{incolor} }]:} \PY{o+ow}{not} \PY{l+m+mi}{1}\PY{o}{\PYZgt{}}\PY{l+m+mi}{2}
\end{Verbatim}


    \hypertarget{variables-and-assignment}{%
\subsubsection{2.1.2 Variables and
Assignment}\label{variables-and-assignment}}

    Variables provide a way to associate names with objects.

In Python, variable names can contain uppercase and lowercase letters,
digits, (but they cannot start with a digit), and the special character
\_. Python variable names are case-sensitive e.g., Julie and julie are
different names.

Finally, there are a small number of reserved words (sometimes called
keywords) in Python that have built-in meanings and cannot be used as
variable names.

In Python, a variable is just a name, nothing more.

    \begin{Verbatim}[commandchars=\\\{\}]
{\color{incolor}In [{\color{incolor}26}]:} \PY{n}{pi} \PY{o}{=} \PY{l+m+mi}{3}
         \PY{n}{radius} \PY{o}{=} \PY{l+m+mi}{11}  \PY{c+c1}{\PYZsh{}  bound to a object of type int}
         \PY{n}{area} \PY{o}{=} \PY{n}{pi} \PY{o}{*} \PY{p}{(}\PY{n}{radius}\PY{o}{*}\PY{o}{*}\PY{l+m+mi}{2}\PY{p}{)}
         \PY{n+nb}{print}\PY{p}{(}\PY{n}{area}\PY{p}{)}
         \PY{n}{radius} \PY{o}{=} \PY{l+m+mf}{14.0} \PY{c+c1}{\PYZsh{} rebound to a different object of type float}
         \PY{n}{area} \PY{o}{=} \PY{n}{pi} \PY{o}{*} \PY{p}{(}\PY{n}{radius}\PY{o}{*}\PY{o}{*}\PY{l+m+mi}{2}\PY{p}{)}
         \PY{n+nb}{print}\PY{p}{(}\PY{n}{area}\PY{p}{)}
\end{Verbatim}


    \begin{Verbatim}[commandchars=\\\{\}]
363
588.0

    \end{Verbatim}

    \textbf{names matter}

Experienced programmers will confirm that they spend a great deal of
time \textbf{reading programs} in an attempt to \textbf{understand} why
they behave as they do.

It is therefore of critical importance to write programs in such way
that they are easy to read.

    \begin{Verbatim}[commandchars=\\\{\}]
{\color{incolor}In [{\color{incolor} }]:} \PY{n}{a} \PY{o}{=} \PY{l+m+mf}{3.14159}
        \PY{n}{b} \PY{o}{=} \PY{l+m+mf}{11.2} 
        \PY{n}{c} \PY{o}{=} \PY{n}{a}\PY{o}{*}\PY{p}{(}\PY{n}{b}\PY{o}{*}\PY{o}{*}\PY{l+m+mi}{2}\PY{p}{)}
\end{Verbatim}


    Apt choice of variable names plays an important role in enhancing
readability.

Another good way to enhance the readability of code is to add comments.

\begin{itemize}
\tightlist
\item
  Text following the symbol \# is not interpreted by Python.
\end{itemize}

    \begin{Verbatim}[commandchars=\\\{\}]
{\color{incolor}In [{\color{incolor}29}]:} \PY{c+c1}{\PYZsh{} subtract area of square s from area of circle c}
         \PY{n}{side}\PY{o}{=}\PY{l+m+mi}{1}
         \PY{n}{areaC} \PY{o}{=} \PY{n}{pi}\PY{o}{*}\PY{n}{radius}\PY{o}{*}\PY{o}{*}\PY{l+m+mi}{2}
         \PY{c+c1}{\PYZsh{} mid}
         \PY{n}{areaS} \PY{o}{=} \PY{n}{side}\PY{o}{*}\PY{n}{side}
         \PY{n}{difference} \PY{o}{=} \PY{n}{areaC}\PY{o}{\PYZhy{}}\PY{n}{areaS}
         \PY{n+nb}{print}\PY{p}{(}\PY{n}{areaS}\PY{p}{)}
\end{Verbatim}


    \begin{Verbatim}[commandchars=\\\{\}]
1

    \end{Verbatim}

    Python allows multiple assignment.

    \begin{Verbatim}[commandchars=\\\{\}]
{\color{incolor}In [{\color{incolor}30}]:} \PY{n}{x}\PY{p}{,} \PY{n}{y} \PY{o}{=} \PY{l+m+mi}{2}\PY{p}{,} \PY{l+m+mi}{3}
\end{Verbatim}


    \begin{Verbatim}[commandchars=\\\{\}]
{\color{incolor}In [{\color{incolor}31}]:} \PY{n}{y}
\end{Verbatim}


\begin{Verbatim}[commandchars=\\\{\}]
{\color{outcolor}Out[{\color{outcolor}31}]:} 3
\end{Verbatim}
            
    \begin{Verbatim}[commandchars=\\\{\}]
{\color{incolor}In [{\color{incolor}32}]:} \PY{n}{x}
\end{Verbatim}


\begin{Verbatim}[commandchars=\\\{\}]
{\color{outcolor}Out[{\color{outcolor}32}]:} 2
\end{Verbatim}
            
    \begin{Verbatim}[commandchars=\\\{\}]
{\color{incolor}In [{\color{incolor}33}]:} \PY{n}{x}\PY{p}{,} \PY{n}{y} \PY{o}{=} \PY{n}{y}\PY{p}{,} \PY{n}{x}   \PY{c+c1}{\PYZsh{}use multiple assignment to swap the bindings of two variables.}
         \PY{n+nb}{print}\PY{p}{(}\PY{l+s+s1}{\PYZsq{}}\PY{l+s+s1}{x =}\PY{l+s+s1}{\PYZsq{}}\PY{p}{,} \PY{n}{x}\PY{p}{)}
         \PY{n+nb}{print}\PY{p}{(}\PY{l+s+s1}{\PYZsq{}}\PY{l+s+s1}{y =}\PY{l+s+s1}{\PYZsq{}}\PY{p}{,} \PY{n}{y}\PY{p}{)}
\end{Verbatim}


    \begin{Verbatim}[commandchars=\\\{\}]
x = 3
y = 2

    \end{Verbatim}

    \hypertarget{the-complex-type}{%
\subparagraph{The complex type}\label{the-complex-type}}

The complexl type is probably not as recognizable to most programmers
because it is not a common built-in data type in other programming
language.

For engineers and scientists, complex numbers are a familiar concept.
Formally, a complex number has a real and an imaginary component, both
represented by float types in Python.

In Python, the imaginary component of a complex number is indicated by a
j

    \begin{Verbatim}[commandchars=\\\{\}]
{\color{incolor}In [{\color{incolor} }]:} \PY{n}{c}\PY{o}{=}\PY{l+m+mi}{1}\PY{o}{+}\PY{l+m+mi}{3}\PY{n}{j}
        \PY{n}{c}
\end{Verbatim}


    \begin{Verbatim}[commandchars=\\\{\}]
{\color{incolor}In [{\color{incolor} }]:} \PY{n}{c}\PY{o}{.}\PY{n}{real}
\end{Verbatim}


    \begin{Verbatim}[commandchars=\\\{\}]
{\color{incolor}In [{\color{incolor} }]:} \PY{n}{c}\PY{o}{.}\PY{n}{imag}
\end{Verbatim}


    \begin{Verbatim}[commandchars=\\\{\}]
{\color{incolor}In [{\color{incolor} }]:} \PY{n}{c}\PY{o}{.}\PY{n}{conjugate}\PY{p}{(}\PY{p}{)}
\end{Verbatim}


    \begin{Verbatim}[commandchars=\\\{\}]
{\color{incolor}In [{\color{incolor} }]:} \PY{n+nb}{abs}\PY{p}{(}\PY{n}{c}\PY{p}{)}
\end{Verbatim}


    \begin{Verbatim}[commandchars=\\\{\}]
{\color{incolor}In [{\color{incolor} }]:} \PY{n}{c1}\PY{o}{=}\PY{n+nb}{complex}\PY{p}{(}\PY{l+m+mi}{4}\PY{p}{,}\PY{l+m+mi}{3}\PY{p}{)}
        \PY{n}{c1}
\end{Verbatim}


    \begin{Verbatim}[commandchars=\\\{\}]
{\color{incolor}In [{\color{incolor} }]:} \PY{n}{c2}\PY{o}{=}\PY{n+nb}{complex}\PY{p}{(}\PY{l+s+s1}{\PYZsq{}}\PY{l+s+s1}{3+4j}\PY{l+s+s1}{\PYZsq{}}\PY{p}{)}
        \PY{n}{c2}
\end{Verbatim}


    \begin{Verbatim}[commandchars=\\\{\}]
{\color{incolor}In [{\color{incolor} }]:} \PY{n}{c3}\PY{o}{=}\PY{n+nb}{complex}\PY{p}{(}\PY{l+s+s1}{\PYZsq{}}\PY{l+s+s1}{3  + 4j}\PY{l+s+s1}{\PYZsq{}}\PY{p}{)}  \PY{c+c1}{\PYZsh{} non blank}
\end{Verbatim}


    \begin{Verbatim}[commandchars=\\\{\}]
{\color{incolor}In [{\color{incolor} }]:} \PY{n}{c1}\PY{o}{+}\PY{n}{c2}
\end{Verbatim}


    \begin{Verbatim}[commandchars=\\\{\}]
{\color{incolor}In [{\color{incolor} }]:} \PY{n}{c1}\PY{o}{\PYZhy{}}\PY{n}{c2}
\end{Verbatim}


    \begin{Verbatim}[commandchars=\\\{\}]
{\color{incolor}In [{\color{incolor} }]:} \PY{n}{c1}\PY{o}{*}\PY{n}{c2}
\end{Verbatim}


    \begin{Verbatim}[commandchars=\\\{\}]
{\color{incolor}In [{\color{incolor} }]:} \PY{n}{c1}\PY{o}{/}\PY{n}{c2}
\end{Verbatim}


    \begin{Verbatim}[commandchars=\\\{\}]
{\color{incolor}In [{\color{incolor} }]:} \PY{n}{c1}\PY{o}{==}\PY{n}{c2}
\end{Verbatim}


    \begin{Verbatim}[commandchars=\\\{\}]
{\color{incolor}In [{\color{incolor} }]:} \PY{n}{c1}\PY{o}{!=}\PY{n}{c2}
\end{Verbatim}


    \begin{Verbatim}[commandchars=\\\{\}]
{\color{incolor}In [{\color{incolor} }]:} \PY{n}{c1}\PY{o}{\PYZgt{}}\PY{n}{c2}
\end{Verbatim}


    \hypertarget{further-reading}{%
\subsubsection{Further Reading}\label{further-reading}}

\begin{itemize}
\item
  Built-in Types:
  https://docs.python.org/3/library/stdtypes.html?highlight=numerical\%20type
\item
  4.4 Numeric Types --- int, float, complex
\item
  探索 Python,第 1 部分: Python 的内置数值类型

  \begin{itemize}
  \tightlist
  \item
    http://www.ibm.com/developerworks/cn/opensource/os-python1/
  \item
    http://www.ibm.com/developerworks/opensource/library/os-python1/?S\_TACT=105AGX52\&S\_CMP=cn-a-os
  \end{itemize}
\end{itemize}

    \hypertarget{idle}{%
\subsubsection{2.1.3 IDLE}\label{idle}}

    Typing programs directly into the shell is highly inconvenient.

Most programmers prefer to use some sort of \textbf{text editor} that is
part of an integrated development environment (IDE).

In this course, we will use IDLE ,the IDE that comes as part of the
standard Python installation package.

When IDLE starts it will open a shell window into which you can type
Python commands

It will also provide you with a file menu and an edit menu

    \hypertarget{branching-programs}{%
\subsection{2.2 Branching Programs}\label{branching-programs}}

    In Python, a conditional statement has the form

    \begin{Verbatim}[commandchars=\\\{\}]
{\color{incolor}In [{\color{incolor} }]:} \PY{k}{if} \PY{n}{Boolean} \PY{n}{expression}\PY{p}{:}
            \PY{n}{block} \PY{n}{of} \PY{n}{code}
        \PY{k}{else}\PY{p}{:}
            \PY{n}{block} \PY{n}{of} \PY{n}{code}
\end{Verbatim}


    \begin{Verbatim}[commandchars=\\\{\}]
{\color{incolor}In [{\color{incolor}38}]:} \PY{n}{x}\PY{o}{=}\PY{l+m+mi}{12}
         \PY{c+c1}{\PYZsh{} the following program that prints “Even” if the value of the variable x is even }
         \PY{c+c1}{\PYZsh{} and “Odd” otherwise:}
         \PY{k}{if} \PY{n}{x} \PY{o}{\PYZpc{}} \PY{l+m+mi}{2} \PY{o}{==} \PY{l+m+mi}{0}\PY{p}{:}
             \PY{n+nb}{print}\PY{p}{(}\PY{l+s+s1}{\PYZsq{}}\PY{l+s+s1}{Even}\PY{l+s+s1}{\PYZsq{}}\PY{p}{)}
             \PY{n+nb}{print}\PY{p}{(}\PY{n}{x}\PY{p}{)}
             \PY{n+nb}{print}\PY{p}{(}\PY{n}{x}\PY{o}{\PYZpc{}}\PY{k}{2})
         \PY{k}{else}\PY{p}{:}
             \PY{n+nb}{print}\PY{p}{(}\PY{l+s+s1}{\PYZsq{}}\PY{l+s+s1}{Odd}\PY{l+s+s1}{\PYZsq{}}\PY{p}{)}
\end{Verbatim}


    \begin{Verbatim}[commandchars=\\\{\}]
Even
12
0

    \end{Verbatim}

    Indentation is semantically meaningful in Python!!!!!!

Python is unusual in using indentation this way. Most other programming
languages use some sort of bracketing symbols to delineate blocks of
code, e.g.,C encloses blocks in braces, \{ \}. An advantage of the
Python approach is that it ensures that the visual structure of a
program is an accurate representation of the semantic structure of that
program.

    \begin{Verbatim}[commandchars=\\\{\}]
{\color{incolor}In [{\color{incolor} }]:} \PY{c+c1}{\PYZsh{}x=2*3*7}
        
        \PY{c+c1}{\PYZsh{}x=2*7}
        
        \PY{n}{x}\PY{o}{=}\PY{l+m+mi}{3}\PY{o}{*}\PY{l+m+mi}{7}
        
        \PY{c+c1}{\PYZsh{} the conditional statements are nested.}
        \PY{k}{if} \PY{n}{x} \PY{o}{\PYZpc{}} \PY{l+m+mi}{2} \PY{o}{==} \PY{l+m+mi}{0}\PY{p}{:}
           
            \PY{k}{if} \PY{n}{x} \PY{o}{\PYZpc{}} \PY{l+m+mi}{3} \PY{o}{==} \PY{l+m+mi}{0}\PY{p}{:}
                \PY{n+nb}{print}\PY{p}{(}\PY{l+s+s1}{\PYZsq{}}\PY{l+s+s1}{Divisible by 2 and 3}\PY{l+s+s1}{\PYZsq{}}\PY{p}{)}
            \PY{k}{else}\PY{p}{:}
                \PY{n+nb}{print}\PY{p}{(}\PY{l+s+s1}{\PYZsq{}}\PY{l+s+s1}{Divisible by 2 and not by 3}\PY{l+s+s1}{\PYZsq{}}\PY{p}{)}
                
        \PY{k}{elif} \PY{n}{x} \PY{o}{\PYZpc{}} \PY{l+m+mi}{3} \PY{o}{==} \PY{l+m+mi}{0}\PY{p}{:}    \PY{c+c1}{\PYZsh{} elif : else  if}
            \PY{n+nb}{print}\PY{p}{(}\PY{l+s+s1}{\PYZsq{}}\PY{l+s+s1}{Divisible by 3 and not by 2}\PY{l+s+s1}{\PYZsq{}}\PY{p}{)}
\end{Verbatim}


    \begin{Verbatim}[commandchars=\\\{\}]
{\color{incolor}In [{\color{incolor} }]:} \PY{n}{x}\PY{o}{=}\PY{l+m+mi}{1}
        \PY{n}{y}\PY{o}{=}\PY{l+m+mi}{10}
        \PY{n}{z}\PY{o}{=}\PY{l+m+mi}{87}
        \PY{k}{if} \PY{n}{x} \PY{o}{\PYZlt{}} \PY{n}{y} \PY{o+ow}{and} \PY{n}{x} \PY{o}{\PYZlt{}} \PY{n}{z}\PY{p}{:}  \PY{c+c1}{\PYZsh{} compound Boolean expressions }
            \PY{n+nb}{print}\PY{p}{(}\PY{l+s+s1}{\PYZsq{}}\PY{l+s+s1}{x is least}\PY{l+s+s1}{\PYZsq{}}\PY{p}{)}
        \PY{k}{elif} \PY{n}{y} \PY{o}{\PYZlt{}} \PY{n}{z}\PY{p}{:}
            \PY{n+nb}{print}\PY{p}{(}\PY{l+s+s1}{\PYZsq{}}\PY{l+s+s1}{y is least}\PY{l+s+s1}{\PYZsq{}}\PY{p}{)}
        \PY{k}{else}\PY{p}{:}
            \PY{n+nb}{print}\PY{p}{(}\PY{l+s+s1}{\PYZsq{}}\PY{l+s+s1}{z is least}\PY{l+s+s1}{\PYZsq{}}\PY{p}{)}
\end{Verbatim}


    \hypertarget{strings-and-input}{%
\subsection{2.3 Strings and Input}\label{strings-and-input}}

Objects of type str are used to represent strings of characters. String
literals are written in a variety of ways: * \textbf{Single quotes}:
'allows embedded ``double'' quotes' * \textbf{Double quotes}: ``allows
embedded 'single' quotes''. * \textbf{Triple quoted}: '`'Three single
quotes''', ``''``Three double quotes''``''

Triple quoted strings may span multiple lines - all associated
whitespace will be included in the string literal.

    \begin{Verbatim}[commandchars=\\\{\}]
{\color{incolor}In [{\color{incolor} }]:} \PY{l+s+s1}{\PYZsq{}}\PY{l+s+s1}{b}\PY{l+s+s1}{\PYZsq{}}
\end{Verbatim}


    \begin{Verbatim}[commandchars=\\\{\}]
{\color{incolor}In [{\color{incolor} }]:} \PY{n+nb}{type}\PY{p}{(}\PY{l+s+s1}{\PYZsq{}}\PY{l+s+s1}{b}\PY{l+s+s1}{\PYZsq{}}\PY{p}{)}
\end{Verbatim}


    \begin{Verbatim}[commandchars=\\\{\}]
{\color{incolor}In [{\color{incolor} }]:} \PY{n+nb}{type}\PY{p}{(}\PY{l+s+s2}{\PYZdq{}}\PY{l+s+s2}{a}\PY{l+s+s2}{\PYZdq{}}\PY{p}{)}
\end{Verbatim}


    \begin{Verbatim}[commandchars=\\\{\}]
{\color{incolor}In [{\color{incolor} }]:} \PY{n}{a}\PY{o}{=}\PY{l+s+s2}{\PYZdq{}\PYZdq{}\PYZdq{}}
        \PY{l+s+s2}{    qqwwqwqwq}
        \PY{l+s+s2}{     qqww}
        \PY{l+s+s2}{     dewewe}
        \PY{l+s+s2}{     ewew}
        \PY{l+s+s2}{     ewew}
        \PY{l+s+s2}{     }
        \PY{l+s+s2}{\PYZdq{}\PYZdq{}\PYZdq{}}
\end{Verbatim}


    \begin{Verbatim}[commandchars=\\\{\}]
{\color{incolor}In [{\color{incolor} }]:} \PY{n}{a}
\end{Verbatim}


    \begin{Verbatim}[commandchars=\\\{\}]
{\color{incolor}In [{\color{incolor} }]:} \PY{n+nb}{print}\PY{p}{(}\PY{n}{a}\PY{p}{)}
\end{Verbatim}


    \begin{Verbatim}[commandchars=\\\{\}]
{\color{incolor}In [{\color{incolor} }]:} \PY{l+m+mi}{3}\PY{o}{*}\PY{l+m+mi}{4}   \PY{c+c1}{\PYZsh{} The operator *}
\end{Verbatim}


    \begin{Verbatim}[commandchars=\\\{\}]
{\color{incolor}In [{\color{incolor} }]:} \PY{l+m+mi}{3}\PY{o}{*}\PY{l+s+s1}{\PYZsq{}}\PY{l+s+s1}{a}\PY{l+s+s1}{\PYZsq{}}   \PY{c+c1}{\PYZsh{} The operator * is overloaded}
        \PY{c+c1}{\PYZsh{} 3*4 is equivalent to 4+4+4, }
        \PY{c+c1}{\PYZsh{} the expression 3*\PYZsq{}a\PYZsq{} is equivalent to \PYZsq{}a\PYZsq{}+\PYZsq{}a\PYZsq{}+\PYZsq{}a\PYZsq{}.}
\end{Verbatim}


    \begin{Verbatim}[commandchars=\\\{\}]
{\color{incolor}In [{\color{incolor} }]:} \PY{l+m+mi}{3}\PY{o}{+}\PY{l+m+mi}{4} \PY{c+c1}{\PYZsh{} The operator *}
\end{Verbatim}


    \begin{Verbatim}[commandchars=\\\{\}]
{\color{incolor}In [{\color{incolor} }]:} \PY{l+s+s1}{\PYZsq{}}\PY{l+s+s1}{a}\PY{l+s+s1}{\PYZsq{}}\PY{o}{+}\PY{l+s+s1}{\PYZsq{}}\PY{l+s+s1}{a1}\PY{l+s+s1}{\PYZsq{}}  \PY{c+c1}{\PYZsh{} The operator * is overloaded}
\end{Verbatim}


    \begin{Verbatim}[commandchars=\\\{\}]
{\color{incolor}In [{\color{incolor} }]:} \PY{n}{a} \PY{o}{=}\PY{l+m+mf}{3.0} 
        \PY{n}{a}\PY{c+c1}{\PYZsh{} name only, not bound to any object, not a literal of any type}
\end{Verbatim}


    \begin{Verbatim}[commandchars=\\\{\}]
{\color{incolor}In [{\color{incolor} }]:} \PY{l+s+s1}{\PYZsq{}}\PY{l+s+s1}{a}\PY{l+s+s1}{\PYZsq{}}\PY{o}{*}\PY{l+s+s1}{\PYZsq{}}\PY{l+s+s1}{a}\PY{l+s+s1}{\PYZsq{}}
\end{Verbatim}


    The type checking in Python is not as, strong as in some other
programming languages (e.g., Java).

    \begin{Verbatim}[commandchars=\\\{\}]
{\color{incolor}In [{\color{incolor} }]:} \PY{l+s+s1}{\PYZsq{}}\PY{l+s+s1}{4}\PY{l+s+s1}{\PYZsq{}} \PY{o}{\PYZlt{}} \PY{l+m+mi}{3}  \PY{c+c1}{\PYZsh{} all numeric values should be less than all values of type str :Python2, not python3!}
\end{Verbatim}


    Strings are one of several sequence types in Python. They share the
following operations with all sequence types.

The length of a string can be found using the len function. For example,
the value of len(`abc') is 3

Indexing can be used to extract individual characters from a string. In
Python,all indexing is zero-based. For example, typing `abc'{[}0{]} into
the interpreter will cause it to display the string `a'.

Slicing is used to extract substrings of arbitrary length. If s is a
string, the expression s{[}start:end{]} denotes the substring of s that
starts at index start and ends at index end-1. For example,
`abc'{[}1:3{]} = `bc'.

    \begin{Verbatim}[commandchars=\\\{\}]
{\color{incolor}In [{\color{incolor} }]:} \PY{n+nb}{len}\PY{p}{(}\PY{l+s+s1}{\PYZsq{}}\PY{l+s+s1}{abd}\PY{l+s+s1}{\PYZsq{}}\PY{p}{)}
\end{Verbatim}


    \begin{Verbatim}[commandchars=\\\{\}]
{\color{incolor}In [{\color{incolor} }]:} \PY{l+s+s1}{\PYZsq{}}\PY{l+s+s1}{abd}\PY{l+s+s1}{\PYZsq{}}\PY{p}{[}\PY{l+m+mi}{0}\PY{p}{]}
\end{Verbatim}


    \begin{Verbatim}[commandchars=\\\{\}]
{\color{incolor}In [{\color{incolor} }]:} \PY{l+s+s1}{\PYZsq{}}\PY{l+s+s1}{abcd}\PY{l+s+s1}{\PYZsq{}}\PY{p}{[}\PY{l+m+mi}{1}\PY{p}{:}\PY{l+m+mi}{3}\PY{p}{]}
\end{Verbatim}


    \begin{Verbatim}[commandchars=\\\{\}]
{\color{incolor}In [{\color{incolor} }]:} \PY{l+s+s1}{\PYZsq{}}\PY{l+s+s1}{abc}\PY{l+s+s1}{\PYZsq{}}\PY{p}{[}\PY{l+m+mi}{0}\PY{p}{:}\PY{n+nb}{len}\PY{p}{(}\PY{l+s+s1}{\PYZsq{}}\PY{l+s+s1}{abc}\PY{l+s+s1}{\PYZsq{}}\PY{p}{)}\PY{p}{]}
\end{Verbatim}


    \begin{Verbatim}[commandchars=\\\{\}]
{\color{incolor}In [{\color{incolor} }]:} \PY{l+s+s1}{\PYZsq{}}\PY{l+s+s1}{abc}\PY{l+s+s1}{\PYZsq{}}\PY{p}{[}\PY{p}{:}\PY{n+nb}{len}\PY{p}{(}\PY{l+s+s1}{\PYZsq{}}\PY{l+s+s1}{abc}\PY{l+s+s1}{\PYZsq{}}\PY{p}{)}\PY{p}{]}  \PY{c+c1}{\PYZsh{} omit the starting index}
\end{Verbatim}


    \hypertarget{advanced-string-slicing}{%
\subsubsection{Advanced String Slicing}\label{advanced-string-slicing}}

you can slice a string with a call such as s{[}i:j{]},

which gives you a portion of string s from index i to index j-1.

However this is not the only way to slice a string!

If you omit the starting index, Python will assume that you wish to
start your slice at index 0.

If you omit the ending index, Python will assume you wish to end your
slice at the end of the string.

    \begin{Verbatim}[commandchars=\\\{\}]
{\color{incolor}In [{\color{incolor} }]:} \PY{l+s+s1}{\PYZsq{}}\PY{l+s+s1}{abc}\PY{l+s+s1}{\PYZsq{}}\PY{p}{[}\PY{p}{:}\PY{p}{]}
\end{Verbatim}


    \begin{Verbatim}[commandchars=\\\{\}]
{\color{incolor}In [{\color{incolor} }]:} \PY{n}{s} \PY{o}{=} \PY{l+s+s1}{\PYZsq{}}\PY{l+s+s1}{Python is Fun!}\PY{l+s+s1}{\PYZsq{}}
        \PY{n}{s}\PY{p}{[}\PY{l+m+mi}{1}\PY{p}{:}\PY{l+m+mi}{5}\PY{p}{]}
\end{Verbatim}


    \begin{Verbatim}[commandchars=\\\{\}]
{\color{incolor}In [{\color{incolor} }]:} \PY{n}{s}\PY{p}{[}\PY{p}{:}\PY{l+m+mi}{5}\PY{p}{]}
\end{Verbatim}


    \begin{Verbatim}[commandchars=\\\{\}]
{\color{incolor}In [{\color{incolor} }]:} \PY{n}{s}\PY{p}{[}\PY{l+m+mi}{1}\PY{p}{:}\PY{p}{]}
\end{Verbatim}


    \begin{Verbatim}[commandchars=\\\{\}]
{\color{incolor}In [{\color{incolor} }]:} \PY{n}{s}\PY{p}{[}\PY{p}{:}\PY{p}{]}
\end{Verbatim}


    That last example is interesting! If you omit both the start and ending
index, you get your original string!

There's one other cool thing you can do with string slicing.

You can add a third parameter, k,

\begin{Shaded}
\begin{Highlighting}[]
\NormalTok{s[i:j:k]}
\end{Highlighting}
\end{Shaded}

This gives a slice of the string s from index i to index j-1, withstep
size k.

Check out the following examples:

    \begin{Verbatim}[commandchars=\\\{\}]
{\color{incolor}In [{\color{incolor} }]:} \PY{n}{s} \PY{o}{=} \PY{l+s+s1}{\PYZsq{}}\PY{l+s+s1}{Python is Fun!}\PY{l+s+s1}{\PYZsq{}}
        \PY{n}{s}\PY{p}{[}\PY{l+m+mi}{1}\PY{p}{:}\PY{l+m+mi}{12}\PY{p}{:}\PY{l+m+mi}{2}\PY{p}{]}
\end{Verbatim}


    \begin{Verbatim}[commandchars=\\\{\}]
{\color{incolor}In [{\color{incolor} }]:} \PY{n}{s}\PY{p}{[}\PY{l+m+mi}{1}\PY{p}{:}\PY{l+m+mi}{12}\PY{p}{:}\PY{l+m+mi}{3}\PY{p}{]}
\end{Verbatim}


    \begin{Verbatim}[commandchars=\\\{\}]
{\color{incolor}In [{\color{incolor} }]:} \PY{n}{s}\PY{p}{[}\PY{p}{:}\PY{p}{:}\PY{l+m+mi}{2}\PY{p}{]}
\end{Verbatim}


    The last example is similar to the example \texttt{s{[}:{]}}.

With \texttt{s{[}::2{]}}, we're asking for the full string s (from index
0 through 13), with a step size of 2

so we end up with every other character in s

    \begin{Verbatim}[commandchars=\\\{\}]
{\color{incolor}In [{\color{incolor} }]:} \PY{n}{s} \PY{o}{=} \PY{l+s+s1}{\PYZsq{}}\PY{l+s+s1}{Python is Fun!}\PY{l+s+s1}{\PYZsq{}}
        \PY{n}{s}\PY{p}{[}\PY{p}{:}\PY{p}{:}\PY{o}{\PYZhy{}}\PY{l+m+mi}{1}\PY{p}{]}  \PY{c+c1}{\PYZsh{}  a step size\PYZlt{}0}
\end{Verbatim}


    \hypertarget{input}{%
\subsubsection{2.3.1 Input}\label{input}}

Python 3 has only one command, input. Python 3's input has the same
semantics as raw\_input in Python 2.7.

input line is treated as a string and becomes the value returned by the
function;

    \begin{Verbatim}[commandchars=\\\{\}]
{\color{incolor}In [{\color{incolor} }]:} \PY{n}{name} \PY{o}{=} \PY{n+nb}{input}\PY{p}{(}\PY{l+s+s1}{\PYZsq{}}\PY{l+s+s1}{Enter your name: }\PY{l+s+s1}{\PYZsq{}}\PY{p}{)}
\end{Verbatim}


    \begin{Verbatim}[commandchars=\\\{\}]
{\color{incolor}In [{\color{incolor} }]:} \PY{n+nb}{print}\PY{p}{(}\PY{l+s+s1}{\PYZsq{}}\PY{l+s+s1}{Are you really}\PY{l+s+s1}{\PYZsq{}}\PY{p}{,} \PY{n}{name}\PY{p}{,} \PY{l+s+s1}{\PYZsq{}}\PY{l+s+s1}{?}\PY{l+s+s1}{\PYZsq{}}\PY{p}{)}  
        \PY{c+c1}{\PYZsh{} print is given multiple arguments, }
        \PY{c+c1}{\PYZsh{} it places a blank space between the values associated with the arguments}
\end{Verbatim}


    \begin{Verbatim}[commandchars=\\\{\}]
{\color{incolor}In [{\color{incolor} }]:} \PY{n+nb}{print}\PY{p}{(}\PY{l+s+s1}{\PYZsq{}}\PY{l+s+s1}{Are you really }\PY{l+s+s1}{\PYZsq{}} \PY{o}{+} \PY{n}{name} \PY{o}{+} \PY{l+s+s1}{\PYZsq{}}\PY{l+s+s1}{?}\PY{l+s+s1}{\PYZsq{}}\PY{p}{)}
        
        \PY{c+c1}{\PYZsh{}  The print statementuses concatenation to produce a string that does not contain }
        \PY{c+c1}{\PYZsh{}  the superfluous blank and passes this as the only argument to print.}
\end{Verbatim}


    \begin{Verbatim}[commandchars=\\\{\}]
{\color{incolor}In [{\color{incolor} }]:} \PY{n}{n} \PY{o}{=} \PY{n+nb}{input}\PY{p}{(}\PY{l+s+s1}{\PYZsq{}}\PY{l+s+s1}{Enter an int: }\PY{l+s+s1}{\PYZsq{}}\PY{p}{)}
\end{Verbatim}


    \begin{Verbatim}[commandchars=\\\{\}]
{\color{incolor}In [{\color{incolor} }]:} \PY{n+nb}{type}\PY{p}{(}\PY{n}{n}\PY{p}{)}
\end{Verbatim}


    Notice that the variable n is bound to the str `3' not the int 3.

    \begin{Verbatim}[commandchars=\\\{\}]
{\color{incolor}In [{\color{incolor} }]:} \PY{n}{n}\PY{o}{*}\PY{l+m+mi}{4}
\end{Verbatim}


    Type conversions (also called type casts) are used often in Python code.
We use the name of a type to convert values to that type.

    \begin{Verbatim}[commandchars=\\\{\}]
{\color{incolor}In [{\color{incolor} }]:} \PY{n}{a}\PY{o}{=}\PY{n+nb}{int}\PY{p}{(}\PY{n}{n}\PY{p}{)}
\end{Verbatim}


    \begin{Verbatim}[commandchars=\\\{\}]
{\color{incolor}In [{\color{incolor} }]:} \PY{n+nb}{type}\PY{p}{(}\PY{n}{a}\PY{p}{)}
\end{Verbatim}


    \begin{Verbatim}[commandchars=\\\{\}]
{\color{incolor}In [{\color{incolor} }]:} \PY{n+nb}{int}\PY{p}{(}\PY{n}{n}\PY{p}{)}\PY{o}{*}\PY{l+m+mi}{4}
\end{Verbatim}


    When a float is converted to an int, the number is truncated (not
rounded), e.g.,

    \begin{Verbatim}[commandchars=\\\{\}]
{\color{incolor}In [{\color{incolor} }]:} \PY{n+nb}{int}\PY{p}{(}\PY{l+m+mf}{3.9}\PY{p}{)}
\end{Verbatim}


    \hypertarget{iteration}{%
\subsection{2.4 Iteration}\label{iteration}}

\begin{Shaded}
\begin{Highlighting}[]

\NormalTok{initial value  }\CommentTok{# !!!}
\ControlFlowTok{while}\NormalTok{ Boolean expression:}
\NormalTok{      block of code}
\end{Highlighting}
\end{Shaded}

    \begin{Verbatim}[commandchars=\\\{\}]
{\color{incolor}In [{\color{incolor} }]:} \PY{c+c1}{\PYZsh{} Square an integer, the hard way X**2}
        \PY{n}{x} \PY{o}{=} \PY{l+m+mi}{4}  
        \PY{n}{ans} \PY{o}{=} \PY{l+m+mi}{0}   
        \PY{n}{itersLeft} \PY{o}{=} \PY{n}{x}      \PY{c+c1}{\PYZsh{} initial value :X}
        
        \PY{k}{while} \PY{p}{(}\PY{n}{itersLeft} \PY{o}{!=} \PY{l+m+mi}{0}\PY{p}{)}\PY{p}{:}
            \PY{n}{ans} \PY{o}{=} \PY{n}{ans} \PY{o}{+} \PY{n}{x}  \PY{c+c1}{\PYZsh{} x**2  to repetitive +}
            \PY{n}{itersLeft} \PY{o}{=} \PY{n}{itersLeft} \PY{o}{\PYZhy{}} \PY{l+m+mi}{1}  
        
        \PY{n+nb}{print}\PY{p}{(}\PY{n+nb}{str}\PY{p}{(}\PY{n}{x}\PY{p}{)} \PY{o}{+} \PY{l+s+s1}{\PYZsq{}}\PY{l+s+s1}{*}\PY{l+s+s1}{\PYZsq{}} \PY{o}{+} \PY{n+nb}{str}\PY{p}{(}\PY{n}{x}\PY{p}{)} \PY{o}{+} \PY{l+s+s1}{\PYZsq{}}\PY{l+s+s1}{ = }\PY{l+s+s1}{\PYZsq{}} \PY{o}{+} \PY{n+nb}{str}\PY{p}{(}\PY{n}{ans}\PY{p}{)}\PY{p}{)}
\end{Verbatim}


    \hypertarget{python-developers-guide}{%
\subsection{2.5 Python Developer's
Guide}\label{python-developers-guide}}

https://www.python.org/dev/

Python Enhancement Proposals

    \hypertarget{the-zen-of-python-pep20-python-enhancement-proposals}{%
\subsubsection{2.5.1 The Zen of Python ,PEP20 (Python Enhancement
Proposals)}\label{the-zen-of-python-pep20-python-enhancement-proposals}}

https://www.python.org/dev/peps/pep-0020/

    \begin{Verbatim}[commandchars=\\\{\}]
{\color{incolor}In [{\color{incolor} }]:} \PY{c+c1}{\PYZsh{} Easter Egg}
        \PY{k+kn}{import} \PY{n+nn}{this}
\end{Verbatim}


    \hypertarget{coding-convention-pep8}{%
\subsubsection{2.5.2 Coding convention:
PEP8()}\label{coding-convention-pep8}}

https://www.python.org/dev/peps/pep-0008/

Use 4 spaces per indentation level.

Limit all lines to a maximum of 79 characters.

Surround top-level function and class definitions with two blank lines.

Code in the core Python distribution should always use UTF-8 (or ASCII
in Python 2).

Imports should usually be on separate lines

Variable, method, modules name should be lower\_case (with underscore,
only if needed)

Constant should be UPPER\_CASE

Class names should normally use the CapWords convention

single letter variable should be limited to loop indexes

    \hypertarget{pep8---python-style-guide-checker}{%
\subsubsection{pep8 - Python style guide
checker}\label{pep8---python-style-guide-checker}}

pep8 is a tool to check your Python code against some of the style
conventions in PEP 8.

\hypertarget{installation}{%
\paragraph{Installation}\label{installation}}

    \begin{Verbatim}[commandchars=\\\{\}]
{\color{incolor}In [{\color{incolor} }]:} \PY{o}{!}pip install pep8
        \PY{o}{!}pip install \PYZhy{}\PYZhy{}upgrade pep8
        \PY{o}{!}pip uninstall pep8
\end{Verbatim}


    \hypertarget{example-usage-and-output}{%
\paragraph{Example usage and output}\label{example-usage-and-output}}

show first occurrence of each error

    \begin{Verbatim}[commandchars=\\\{\}]
{\color{incolor}In [{\color{incolor}1}]:} \PY{o}{!}pep8 \PYZhy{}\PYZhy{}first ./src/ch3\PYZus{}cube\PYZus{}root.py
\end{Verbatim}


    \begin{Verbatim}[commandchars=\\\{\}]
./src/ch3\_cube\_root.py:2:2: E225 missing whitespace around operator
./src/ch3\_cube\_root.py:3:8: W291 trailing whitespace
./src/ch3\_cube\_root.py:5:1: W293 blank line contains whitespace
./src/ch3\_cube\_root.py:7:16: E114 indentation is not a multiple of four (comment)
./src/ch3\_cube\_root.py:7:16: E116 unexpected indentation (comment)
./src/ch3\_cube\_root.py:15:28: E231 missing whitespace after ','
./src/ch3\_cube\_root.py:15:39: W292 no newline at end of file

    \end{Verbatim}

    You can also make pep8.py show the source code for each error, and even
the relevant text from PEP 8:

    \begin{Verbatim}[commandchars=\\\{\}]
{\color{incolor}In [{\color{incolor} }]:} \PY{o}{!}pep8 \PYZhy{}\PYZhy{}show\PYZhy{}source \PYZhy{}\PYZhy{}show\PYZhy{}pep8 ./src/ch3\PYZus{}cube\PYZus{}root.py
\end{Verbatim}


    you can display how often each error was found:

    \begin{Verbatim}[commandchars=\\\{\}]
{\color{incolor}In [{\color{incolor} }]:} \PY{o}{!}pep8 \PYZhy{}\PYZhy{}statistics ./src/ch3\PYZus{}cube\PYZus{}root.py
\end{Verbatim}


    \hypertarget{autopep8}{%
\subsubsection{autopep8}\label{autopep8}}

A tool that automatically formats Python code to conform to the PEP 8
style guide

    \begin{Verbatim}[commandchars=\\\{\}]
{\color{incolor}In [{\color{incolor} }]:} \PY{o}{!}pip install autopep8
\end{Verbatim}


    To modify a file in place (with aggressive level 2):

    \begin{Verbatim}[commandchars=\\\{\}]
{\color{incolor}In [{\color{incolor} }]:} \PY{o}{!}autopep8 \PYZhy{}\PYZhy{}in\PYZhy{}place \PYZhy{}\PYZhy{}aggressive \PYZhy{}\PYZhy{}aggressive ./src/ch3\PYZus{}cube\PYZus{}root.py
\end{Verbatim}


    \hypertarget{pylint}{%
\section{Pylint}\label{pylint}}

Pylint is a tool that checks for errors in Python code, tries to enforce
a coding standard and looks for bad code smells

http://www.pylint.org

    \begin{Verbatim}[commandchars=\\\{\}]
{\color{incolor}In [{\color{incolor} }]:} \PY{o}{!}pip install pylint
\end{Verbatim}


    \hypertarget{google-python-style-guide}{%
\section{Google Python Style Guide}\label{google-python-style-guide}}

Every major open-source project has its own style guide: a set of
conventions (sometimes arbitrary) about how to write code for that
project. It is much easier to understand a large codebase when all the
code in it is in a consistent style.

``Style'' covers a lot of ground, from ``use camelCase for variable
names'' to ``never use global variables'' to ``never use exceptions.''
This project holds the style guidelines we use for Google code. If you
are modifying a project that originated at Google, you may be pointed to
this page to see the style guides that apply to that project.

http://google.github.io/styleguide/pyguide.html

中文: https://github.com/zh-google-styleguide/zh-google-styleguide

    \hypertarget{python-2-to-3-code-translation}{%
\section{Python 2 to 3 code
translation}\label{python-2-to-3-code-translation}}

The Python Library Reference: 26.7 2to3 - Automated Python 2 to 3 code
translation

2to3 is a Python program that reads Python 2.x source code and applies a
series of fixers to transform it into valid Python 3.x code.

2to3 will usually be installed with the Python interpreter as a script.
It is also located in the Tools/scripts directory of the Python root.

Here is a sample Python 2.x source file, example.py, It can be converted
to Python 3.x code via 2to3 on the command line:

python C:/Python35/Tools/Scripts/2to3.py -w example.py


    % Add a bibliography block to the postdoc
    
    
    
    \end{document}
